\chapter{Kriptografija}
\label{ch:crypto}

Kriptografija je znanost koja omogućava sigurnu komunikaciju. Način ostvarivanja
sigurne komunikacije ovisi o sigurnosnim zahtjevima koje komunikacija mora
zadovoljavati. Sigurnosni zahtjevi osiguravaju se korištenjem kriptografskih
algoritama koji određene matematičke principe primjenjuju za ostvarivanje tih
zahtjeva. Kriptografija omogućava ostvarivanje sljedećih sigurnosnih zahtjeva:
tajnosti, cjelovitosti, autentifikacije i neporecivosti.

Tajnost predstavlja komunikaciju koja je poznata samo sudionicima komunikacije.
Cjelovitost osigurava da su podaci nepromijenjeni prilikom prijenosa.
Autentifikacija predstavlja mogućnost dokazivanja identiteta tijekom
komunikacije. Neporecivost daje mogućnost dokaza da je jedan od sudionika obavio
određenu radnju.

\section{Kriptografski algoritmi}

Kriptografske algoritme možemo podijeliti u tri osnovne skupine:
\begin{itemize}
\item algoritmi za stvaranje kriptografskih sažetaka (engl. \emph{cryptographic hash}),
\item algoritmi simetričnog kriptosustava (engl. \emph{symmetric cryptosystem}) i
\item algoritmi asimetričnog kriptosustava (engl. \emph{asymmetric
    cryptosystem}).
\end{itemize}

U nastavku slijede kratki opisi navedenih vrsta kriptografskih algoritama
uključujući i njihovu primjenu u osiguravanju sigurnosnih zahtjeva.

\section{Kriptografski sažetak}

Funkcija za stvaranje kriptografskog sažetka jednosmjerna je funkcija koja
sažima niz podataka
proizvoljne veličine u niz podataka točno određene veličine. Kriptografski
sažetak najčešće se koristi za provjeru cjelovitosti podataka.
Da bi se funkcija $h$ smatrala jednosmjernim kriptografskim sažetkom (engl.
\emph{one-way hash function}), treba zadovoljavati sljedeća svojstva \cite[str.
544]{van2011encyclopedia}:
\begin{enumerate}
\item Argument $X$ može biti proizvoljne duljine, a rezultat $h(X)$ je fiksne
duljine od $n$ okteta. Izračunavanje sažetka se uobičajeno zapisuje:
$$h(X) = Y;\ \mathit{duljina}(Y) = n$$
\item Funkcija sažetka $h$ mora biti jednosmjerna na način da je za ponuđeni
$Y$,
računalno neisplativo tražiti poruku $X$ za koju vrijedi $h(X)=Y$ (engl.
\emph{preimage resistant}).
\item Također, za danu poruku $X$ i $h(X)$ mora biti
računalno neisplativo tražiti $X' \neq X$ takav da je $h(X') = h(X)$ (engl.
\emph{second preimage resistant}).
\item Poželjno svojstvo jednosmjernih funkcija sažetaka je otpornost na
podudaranje (engl. \emph{collision resistant}), odnosno da je računalno
teško (i neisplativo) pronaći bilo koje dvije različite poruke koje daju isti
sažetak.
\end{enumerate}

Kriptografski sažetak se koristi za razne primjene: brzo uspoređivanje datoteka,
adresiranje podataka u programiranju, spremanje podataka za autentifikaciju
korisnika i provjeru cjelovitosti. Ako se provjera cjelovitosti koristi prilikom
komunikacije, kriptografski sažetak se mora dodatno zaštititi
kako ga napadač ne bi mogao promijeniti.
Dodatna zaštita osigurava se korištenjem asimetrične kriptografije, a rezultat
je digitalni potpis.

Sigurna provjera cjelovitosti može se obaviti i isključivo korištenjem
kriptografskog sažetka ako komunicirajuće strane imaju na raspolaganju
dijeljenu zajedničku tajnu. Zajednička tajna se tada uključuje u izračun
sažetka kako napadač ne bi mogao promijeniti poruku i izračunati novi sažetak za
provjeru cjelovitosti i autentičnosti poruke. Ova metoda izračunavanja naziva se
izračun zaštitne sume korištenjem algoritama kriptografskog sažetka (engl.
\emph{Hashed Message Authentication Code} - HMAC) i pobliže je opisana u
potpoglavlju \ref{ssec:HMAC}.

Postoji velika količina algoritama za kriptografski sažetak, ali se
za osiguravanje komunikacije najviše koriste MD5 (engl. \emph{Message Digest}) i
SHA (engl. \emph{Secure Hash Algorithm}) algoritmi.
Ako algoritam sažetka nije ranjiv na kriptografske napade, duljina
sažetka dodatno povećava sigurnost algoritma jer povećava prostor mogućih
vrijednosti
sažetka i smanjuje mogućnost podudaranja sažetaka. Duljine navedenih sažetaka
mogu biti sljedeće:
\begin{itemize}
\item MD5 - 128 bita,
\item SHA1 - 160 bita,
\item SHA2/SHA3 - 224, 256, 384 ili 512 bita.
\end{itemize}

\subsection{HMAC}
\label{ssec:HMAC}

HMAC (\emph{Hashed Message Authentication Code}) je poseban slučaj algoritma MAC
(\emph{Message Authentication Code}) koji služi za izračun podatka za
autentifikaciju poruka u komunikaciji korištenjem kriptografskog sažetka.
Algoritam MAC iz poruke $X$ i ključa $K$ računa autentifikacijski kod $A$:
$$MAC_K(X)=A$$
Za generiranje takvog podatka mogu se koristiti različite metode, ali najbolji
rezultati u praksi se postižu korištenjem kriptografskih funkcija
\cite[str. 746]{van2011encyclopedia}.
\nomenclature{MAC}{\emph{Message Authentication Code}}
\nomenclature{HMAC}{\emph{Hashed Message Authentication Code}}

Postupak kreiranja rezultata HMAC s ključem $K$ je za poruku $X$
definiran je sljedećim izrazom \cite[str. 559]{van2011encyclopedia}:
$$HMAC_K(X)=h(\ (K \oplus opad)\ ||\ h( (K \oplus ipad) || X )\ )$$
Konstante $opad$ i
$ipad$ su nizovi okteta duljine ključa $K$, znak $||$ predstavlja operaciju
konkatenacije nizova, a $\oplus$ operaciju isključivo ili (XOR). $opad$ je niz
okteta sadržaja \texttt{0x36}, dok je $ipad$ niz okteta sadržaja \texttt{0x5C}.
Algoritam HMAC može koristiti bilo koji algoritam za stvaranje kriptografskog
sažetka prilikom izračuna rezultata.

S obzirom na to da HMAC koristi jedan ključ za sve poruke koje se razmjenjuju u
komunikaciji, njegovim korištenjem ne osigurava se neporecivost poruke, ali HMAC
predstavlja vrlo brz način izračunavanja podataka za provjeru cjelovitosti
poruke i autentifikaciju
pošiljatelja, zato što je stvaratelj poruke $X$ i $HMAC_K(X)$ morao biti upoznat
s ključem $K$.

Kako bi se određeni ključ $K$ mogao koristiti za HMAC operacije on mora biti
jednak duljini rezultata algoritma za izračunavanje sažetka. Primjerice, ako se
koristi algoritam MD5 tajni ključ mora biti duljine barem 128 bita, a ako se
koristi algoritam SHA-256 tada bi ključ trebao biti duljine barem 256 bita.  

\section{Simetrični kriptosustav}
Simetrični kriptosustav označava sustav u kojem mora postojati jedan ključ s
kojim se odvijaju operacije u sustavu. Ključ koji se
koristi za izvođenje operacija je zajednička tajna između komunicirajućih
strana. Simetrični kriptosustav za razliku od kriptografskog sažetka mora imati
ključ, dok se kriptografski sažetak može računati bez ključa ili s ključem u
slučaju HMAC-a.

Cilj simetričnog kriptosustava je osiguravanje tajnosti komunikacije,
stoga su definirane dvije osnovne operacije u sustavu: šifriranje podataka u
šifrat i dešifriranje šifrata u izvorne podatke. Operacije šifriranja poruke $X$
u šifrat $Y$ i dešifriranja šifrata u poruku korištenjem ključa $K$ definirane su
sljedećim izrazima:
$$E_K(X)=Y;\ D_K(Y)=X$$

Simetrični kriptosustavi dijele se u dvije osnovne skupine s obzirom na način
obrade podataka. Postoje algoritmi koji šifriraju tokove podataka  (engl.
\emph{stream ciphers}) i algoritmi koji šifriraju blokove podataka (engl.
\emph{block ciphers}).

\subsection{Šifriranje tokova podataka}
Kod šifriranja tokova podataka niz znakova poruke $m_0,m_1,m_2,...$ šifrira se u
niz
znakova šifrata $c_0,c_1,c_2,...$ na sljedeći način. Pseudo-slučajan niz
znakova $s_0,s_1,s_2,...$ generira se automatom čiji je ulaz tajni ključ.
Znak u nizu $s_i$ ovisi o tajnom ključu i o prethodnom znaku
$s_{i-1}$ dok znak $s_0$ ovisi o tajnom ključu i nekom javno definiranom
parametru. Znak šifrata $c_j$ izračunava se
kombiniranjem znaka poruke $m_j$ i
znaka $s_j$ iz pseudo-slučajnog niza.

Šifriranje tokova podataka najčešće se koristi u komunikaciji s puno smetnji
kako bi se šifrat mogao prenositi u što manjim dijelovima tako da gubitak manjeg
dijela podataka ne uzrokuje gubitak cijelog bloka podataka, kao što je to slučaj
kod šifriranja blokova. Još jedna važna primjena je u sustavima koji zahtijevaju
veliku brzinu prijenosa podataka jer se svaki znak šifrata može dešifrirati
nezavisno čim dođe u sustav \cite[str. 1265]{van2011encyclopedia}.

Algoritmi za šifriranje tokova podataka stvaraju se za specifične namjene,
a najčešće korišteni algoritmi su RC4 za osiguravanje tokova pri prijenosu
podataka u računalnim mrežama i A5/1 za osiguravanje
podataka u mobilnim GSM mrežama.

\subsection{Šifriranje blokova podataka}

Algoritam za šifriranje blokova podataka za dani ključ $K$ definira
šifriranje kao izračunavanje \emph{n} bita šifrata iz \emph{n} bita poruke, te
dešifriranje
kao izračunavanje \emph{n} bita poruke iz \emph{n} bita šifrata. Algoritam za
šifriranje
blokova trebao bi zadovoljavati dva osnovna svojstva: zbrku (engl.
\emph{confusion}) i raspršenost (engl. \emph{diffusion}). Zbrka kod šifriranja
predstavlja minimalnu statističku vezu između poruke i šifrata dok raspršenost
uvjetuje da bi svaki bit poruke trebao utjecati na što veći dio šifrata.
\cite[str. 152]{van2011encyclopedia}

Šifriranje blokova podataka koristi se za osiguravanje tajnosti u komunikaciji,
ali isto tako i za sigurnu pohranu podataka u sklopu računalnih sustava.
Najzastupljeniji algoritmi za šifriranje blokova podataka su DES (engl.
\emph{Data Encryption Standard}) \cite[str. 295]{van2011encyclopedia} i AES (engl. \emph{Advanced
Encryption Standard}) \cite[str.1046]{van2011encyclopedia}
\cite{daemen2013design}. Algoritam DES predstavlja stari standard koji
se koristi u sustavima koji su dizajnirani u prošlosti. Algoritam AES je brži i
sigurniji nasljednik algoritma DES. Osnovne značajke tih algoritama su sljedeće:
\begin{itemize}
\item DES šifrira blokove veličine 64 bita s ključem duljine 56 bita.
\item AES šifrira blokove veličine 128-bita s ključevima duljine 128, 192 ili
256 bita.
\end{itemize}
Što je veličina ključa veća algoritam je u pravilu sigurniji jer je prostor u
kojem se generiraju ključevi veći i teži za pretražiti od strane napadača.
Veličina ključa izravno utječe i na brzinu šifriranja, odnosno dešifriranja. Što
je ključ veći to je brzina manja.

Algoritmi za šifriranje blokova predstavljaju prilagodljivi primitiv za
ostvarivanje raznovrsnih namjena, osim šifriranja podataka, poput
autentificiranog šifriranja (engl. \emph{authenticated encryption}), MAC
algoritama i funkcija kriptografskog sažetka. Algoritmi šifriranja blokova
mogu se koristiti u različitim načinima rada kako bi mogli biti zadovoljeni
različiti sigurnosni zahtjevi.

\subsubsection{Načini šifriranja blokova}
Vrlo važan parametar kod šifriranja blokova je način primjene algoritma
šifriranja blokova. Način rada algoritama za šifriranje blokova
određuje na koji će se način šifrirani blokovi spajati u konačni šifrat. Odabir
načina rada izravno utječe na sigurnost i brzinu šifriranja blokova \cite[str.
790]{van2011encyclopedia}. Najznačajniji i najčešće korišteni načini rada su CBC
i CTR. Opis načina rada CBC i CTR prikazan je na primjeru algoritma AES koji je
trenutni
standard za simetrično šifriranje blokova podataka. Dok CBC način rada omogućuje
samo šifriranje blokova podataka, CTR i drugi srodni načini rada mogu se
koristiti za šifriranje tokova podataka zbog toga što iz tajnog ključa
generiraju pseudo-slučajan niz znakova koji se mogu koristiti za šifriranje
tokova podataka.

CBC (engl. \emph{Cipher Block Chaining}) koristi ulančavanje
blokova na način da se šifrat prošlog bloka bit po bit zbraja po modulu 2 s
trenutnom porukom prije šifriranja te poruke. Početni dio poruke zbraja se s
promjenjivom početnom vrijednosti koja se označava s IV (engl.
\emph{Initialization Vector}). Dodavanje operacije zbrajanja poruke u aritmetici
modulo 2 omogućuje veću količinu raspršenosti podataka u odnosu na
najjednostavniji ECB (engl. \emph{Electronic CodeBook}) način rada. Blokovi
postaju međusobno ovisni ali samo u okviru definiranog prozora. Prozor
predstavlja niz određenog broja blokova koji izravno ili neizravno utječu jedni
na druge. U CBC
načinu rada odnos između poruke $p_i$, ključa $K$ i šifrata $c_i$ definiran je
sljedećim izrazima:
$$c_i=E_K(p_i \oplus c_{i-1});\ p_i=D_K(c_i \oplus c_{i-1})$$

CTR (engl. \emph{CounTeR mode}) način rada omogućuje potpunu nezavisnost između
blokova, ali se za svaki blok koji se šifrira koristi dodatan podatak čija je
svrha povećanje raspršenosti prilikom kodiranja. Kako su svi blokovi međusobno
neovisni mogu se paralelno obrađivati što olakšava brzo izračunavanje šifrata iz
poruke i obrnuto. Koristi se dodatni podatak stvoren iz jedinstvene jednokratne
vrijednosti (engl. \emph{nonce}), početne vrijednosti IV i brojača koji kreće od
broja jedan. Za svaki blok vrijednost tog podatka uvećava se za jedan
\cite{rfc3686}. Izračun šifrata $c_i$ iz poruke $p_i$ uz ključ $K$ i dodatnu
vrijednost CTR bloka $CTRBLK$ definiran je sa sljedećim izrazima:
$$CTRBLK = nonce || IV || i$$
$$c_i = p_i \oplus E_K(CTRBLK)$$
$$p_i = c_i \oplus E_K(CTRBLK)$$
gdje $||$ označava operaciju konkatenacije nizova okteta.

CBC način rada se smatra standardom, ali CTR ostvaruje veću
brzinu i sigurnost uz redovito osvježavanje i izbjegavanje ponovnog korištenja
istih dodatnih podataka.

\subsection{Autentificirano šifriranje}
Autentificirano šifriranje (engl. \emph{Authenticated Encryption} - AE) omogućava
da se tijekom izračuna šifrata izračuna i autentifikacijski podatak (MAC) koji
će potvrditi da šifrat nije mijenjan prilikom prijenosa podataka. Kao osnova za
AE koristi se algoritam za simetrično šifriranje blokova. Šifriranje se
odrađuje s jednim od prethodno navedenih načina rada (CBC, CTR), ali postoje različiti
pristupi izračuna autentifikacijskih podataka. Autentifikacijski podaci se, u
pravilu, izračunavaju uz pomoć funkcije kriptografskog sažetka i danog ključa
(MAC, odnosno HMAC). Za svaku operaciju potrebno je koristiti različite
ključeve $K_1$ za autentifikaciju i $K_2$ za šifriranje. Postoje sljedeći načini
izračunavanja AE podataka:
\begin{itemize}
\item Autentificiraj pa šifriraj (engl. \emph{Mac-then-Encrypt} - MtE) - iz izvorne
poruke $M$ prvo se generiraju podaci za autentifikaciju pa se zatim šifrira poruku
zajedno s autentifikacijskim podacima.
$$MtE=E_{K_2}(M,MAC_{K_1}(M))$$
\item Šifriraj pa autentificiraj (engl. \emph{Encrypt-then-Mac} - EtM) - prvo se
izvornu poruku šifrira pa se podaci za autentifikaciju stvaraju iz šifrata i
šalju zajedno.
$$C=E_{K_2}(M);\ EtM=C,MAC_{K_1}(C)$$
\item Šifriraj i autentificiraj (engl. \emph{Encrypt-and-Mac}) - odvojeno se
šifrira poruka i izračunavaju autentifikacijski podaci iz poruke i šalju
zajedno.
$$E\&M=E_{K_2}(M),MAC_{K_1}(M)$$
\end{itemize}

Trenutno se u praksi koriste AE programska ostvarenja čija je osnova algoritam
šifriranja
blokova AES, AES-CCM \cite{rfc6655} i AES-GCM \cite{rfc5288}. AES-CCM koristi
način rada CBC i izračunavanje MtE, dok AES-GCM koristi način rada CTR i
izračunavanje EtM. Oboje su se pokazali dovoljno sigurnim za primjenu, ali zbog
pronađenih ranjivosti izračunavanja EtM savjetuje se koristiti MtE za sigurnu
komunikaciju \cite{rfc7366}.

AE se smatra standardom za sigurno slanje podataka jer osigurava tri sigurnosna
zahtjeva odjednom: tajnost, integritet i autentifikaciju, što predstavlja ključ
sigurne komunikacije, jer nije dovoljno samo tajno razmjenjivati podatke, već i
utvrditi da podaci nisu mijenjani tijekom prijenosa
\cite[str. 54]{van2011encyclopedia}.

\section{Asimetrični kriptosustav}
Asimetrični kriptosustav koristi različite ključeve za šifriranje i dešifriranje
podataka. Ako se podatak šifrira s jednim ključem može se dešifrirati samo s
odgovarajućim drugim
ključem i obratno. U ovom sustavu moguće je jedan ključ učiniti javnim bez
ugrožavanja tajnosti drugog ključa \cite[str. 49]{van2011encyclopedia}.

Asimetrično šifriranje se općenito može definirati sljedećim izrazima, gdje $E$
označava šifriranje, $D$ dešifriranje, $PK$ javni ključ, $SK$ tajni ključ, $M$
poruku, a $C_1$ i $C_2$ šifrate:
$$E_{PK}(M)=C_1;\ D_{SK}(C_1)=M$$
$$E_{SK}(M)=C_2;\ D_{PK}(C_2)=M$$

U asimetričnom kriptosustavu svaki tajni ključ ima odgovarajući javni
ključ. Asimetrični kriptosustavi mogu se primjenjivati za razne namjene poput
digitalnog potpisa, digitalne omotnice i dogovaranja zajedničkog tajnog ključa
između komunicirajućih strana. U ovom sustavu također vrijedi da je sigurnost
veća što je veća duljina ključa.
Za postizanje iste razine sigurnosti, u asimetričnom kriptosustavu se
koriste ključevi veće duljine u odnosu na ključeve korištene u simetričnom
kriptosustavu.

\subsection{Digitalni potpis}
U slučaju da se podatak šifrira tajnim ključem, svi koji su u posjedu
odgovarajućeg javnog ključa mogu doći do tih podataka. Navedeni postupak služi
za osiguravanje neporecivosti, autentifikacije i integriteta poruke te
predstavlja digitalni potpis. Neporecivost je moguće ostvariti jer svaka strana
u komunikaciji posjeduje tajni ključ koji mora biti poznat samo toj strani i
nikome drugome.

Kako se digitalnim potpisom osigurava integritet poruke nije potrebno šifrirati
cjelokupni sadržaj poruke, već je dovoljno šifrirati samo njezin sažetak.
Digitalni potpis stvara se korištenjem kriptografskog sažetka, čiji se
rezultat šifrira tajnim ključem pošiljatelja. Stvaranje digitalnog potpisa
poruke $M$
korištenjem funkcije sažetka $h$ korištenjem tajnog ključa $SK$ opisano je
sljedećim izrazom:
$$\mathit{digitalni\_potpis} = E_{SK}(h(M))$$
Digitalni potpis se uvijek šalje uz odgovarajuću poruku, kako bi se mogao
provjeriti integritet i autentičnost poruke. Po primitku poruke primatelj uzima
odgovarajući javni ključ $PK$ kako bi dešifrirao primljeni izvorni sažetak
poruke $S_1$ i
usporedio taj sažetak s izračunatim sažetkom $S_2$:
$$S_1 = E_{PK}(\mathit{digitalni\_potpis});\ S_2=h(M)$$

U slučaju da su sažeci $S_1$ i $S_2$ jednaki znamo da je poruku potpisao vlasnik
tajnog
ključa koji odgovara korištenom javnom ključu i da poruka nije mijenjana tijekom
prijenosa.

\nomenclature{RSA}{Rivest Shamir Alderman asimetrični algoritam}
\nomenclature{ECDSA}{\emph{Elliptic Curve Digital Signature Algorithm}}
Za digitalni potpis najzastupljeniji algoritmi su DSA(engl. \emph{Digital
Signature Algorithm}), RSA (Rivest Shamir Alderman) i ECDSA (engl.
\emph{Elliptic Curve DSA}) . Algoritam
DSA \cite{gallagher2013fips} koristi ElGamalov
način stvaranja digitalnog potpisa \cite[str. 395]{van2011encyclopedia} koji se temelji na
problemu diskretnog logaritma \cite[str. 352]{van2011encyclopedia}. U sklopu
standarda definirane su tri veličine ključa, 1024, 2048 i 3072 bita
\cite{gallagher2013fips}.

Algoritam RSA temelji se na problemu faktorizacije
velikih prostih brojeva i može koristiti ključeve različitih veličina. Algoritmi
DSA i RSA pružaju usporedivu razinu sigurnosti za istu duljinu
tajnog ključa. RSA koristi ključeve duljine 1024, 2048 i 4096 bitova.  Razina
sigurnosti raste s veličinom ključa, ali brzina šifriranja/dešifriranja pada.

\subsection{Digitalna omotnica}
Ako se sadržaj poruke šifrira javnim ključem do izvorne poruke može doći samo
strana
koja posjeduje odgovarajući tajni ključ. Taj postupak služi za ostvarivanje
tajnosti u komunikaciji i predstavlja digitalnu omotnicu.

Kako je postupak asimetričnog šifriranja značajno sporiji od postupka simetričnog
šifriranja ta se dva postupka zajednički primjenjuju u hibridnom načinu
šifriranja. U hibridnom načinu šifriranja poruka se šifrira simetričnim ključem
$K$, a taj simetrični ključ javnim ključem $PK$ primatelja podataka kako bi
samo on mogao doći do ključa, odnosno sadržaja šifrirane poruke $M$:
$$DO_1 = E_K(M);\ DO_2 = E_{PK}(K)$$
$$digitalna\_omotnica = DO_1, DO_2$$

Za otvaranje digitalne omotnice, odnosno dešifriranje izvorne poruke primatelju je
dovoljan tajni ključ $SK$ koji odgovara javnom ključu $PK$ s kojim je
stvorena omotnica:
$$K = D_{SK}(DO_2);\ M = D_{K}(DO_1)$$

Za stvaranje digitalne omotnice koristi se algoritam RSA koji je opisan u
prethodnom potpoglavlju.

\subsection{Dogovaranje zajedničke tajne}
\label{sec:dh}
Ako se komunicirajuće strane prethodno ne poznaju te ne posjeduju tajne i
javne ključeve za zaštitu komunikacije, one mogu dogovoriti zajedničku tajnu
koja će se koristiti za zaštitu komunikacije. Protokoli za dogovaranje
zajedničke tajne koriste tajnu vrijednost i odgovarajuću javnu vrijednost za
dogovor, te predstavljaju poseban slučaj asimetričnog kriptosustava.

Većina algoritama za dogovaranje zajedničke tajne definirana je u
protokolu Diffie-Hellman \cite{DH} (DH), a temelji se na problemu diskretnog
logaritma. Postupak izračuna pobliže je opisan u nastavku.

\nomenclature{DH}{protokol Diffie-Hellman}

Kako bi komunicirajuće strane mogle dogovoriti zajedničku tajnu moraju dijeliti 
iste početne vrijednosti za izračun. To su:
\begin{itemize}
\item prosti broj $p$ i
\item generator $1 \leq g \leq p-1$ takav da je $g^{p-1} \equiv 1(mod p)$.
\end{itemize}

Neka su započinjatelj (I) i primatelj (R) dvije strane koje žele dogovoriti
zajedničku tajnu. Tijek komunikacije je sljedeći:
\begin{enumerate}
\item I generira nasumičnu vrijednost $1 \leq x \leq p-2$ i izračunava $z_x=g^x mod p$.
Rezultat operacije $z_x$ šalje strani R.
\item R generira nasumičnu vrijednost $1 \leq y \leq p-2$ i izračunava $z_y=g^y mod p$.
Rezultat operacije $z_y$ šalje strani I.
\item Nakon primitka $z_y$, I izračunava $K_I = {z_y}^x mod p$.
\item Nakon primitka $z_x$, R izračunava $K_R = {z_x}^y mod p$.
\end{enumerate}

Vrijednosti $K_I$ i $K_R$ su jednake te su time strane I i R
uspješno dogovorile zajedničku tajnu. Vrijednosti $x$ i $y$ predstavljaju tajne
ključeve u razmjeni dok su $z_x$ i $z_y$ javni ključevi. Napadač prisluškivanjem
komunikacije ima pristup vrijednostima $p$, $g$, $z_x$ i $z_y$. Pronalazak
dogovorene tajne je ekvivalentan pronalaženju jednog tajnog ključa. U slučaju
dobro odabranih parametara to je težak problem. Ako su ključevi
dobro generirani i dovoljno veliki pasivni napadač nije u mogućnosti doći do
zajedničke tajne. \cite[str. 341]{van2011encyclopedia}

\subsection{Algoritmi temeljeni na eliptičnim krivuljama}

Nad skupom točaka eliptične krivulje mogu se definirati sve matematičke
operacije potrebne za izvođenje kriptografskih operacija poput onih temeljenih
na problemu diskretnog algoritma.
Algoritmi koji koriste eliptične krivulje
koriste manje kriptografske ključeve u odnosu na standardne algoritme.
Posljedica toga je brži izračun kriptografskih operacija za jednaku
razinu sigurnosti u odnosnu na standardne algoritme.
Algoritmi temeljeni na eliptičnim krivuljama mogu se koristiti za digitalno
potpisivanje i dogovor zajedničke tajne.
\cite[str. 397]{van2011encyclopedia}

\nomenclature{EC}{kriptografski algoritam temeljen na eliptičnim krivuljama}

Algoritam ECDSA \cite[str. 407]{van2011encyclopedia} je inačica algoritma DSA
koja umjesto diskretnih algoritama za
temelj koristi eliptične krivulje. Prednost eliptičnih krivulja je
manja veličina tajnog ključa za postizanje jednake razine sigurnosti. Algoritam
RSA s
ključem duljine 2048 daje podjednaku razinu sigurnosti kao algoritam ECDSA s
veličinom ključa
od 224 bita.

ECDH \cite[str. 400]{van2011encyclopedia} (engl. \emph{Elliptic Curve DH}) je
inačica protokola DH koja koristi
eliptične krivulje za izračunavanje zajedničke tajne. Glavne prednosti su
smanjenje veličine ključeva uz zadržavanje iste razine sigurnosti te
jednostavnije i brže generiranje tajnih ključeva.
Iako algoritmi temeljeni na eliptičnim krivuljama pružaju mnoge prednosti u vidu
manje veličine ključeva i bržeg izračuna, potrebno je odabrati pouzdanu
eliptičnu krivulju za zadovoljavajuću razinu sigurnosti \cite{safecurves}.

\section{Autentificirano dogovaranje zajedničke tajne}

Protokoli za dogovor zajedničke tajne ranjivi su na napad čovjek u sredini
(engl. \emph{Man-in-the-Middle attack} - MITM). Ako se napadač nađe između dvije
komunicirajuće strane on može neovisno dogovoriti zajedničku tajnu sa svakom
stranom i doći do svih podataka koje te dvije strane razmjenjuju. Tijek napada
na protokol Diffie-Hellman (DH) je sljedeći:
\begin{enumerate}
\item Započinjatelj (I) generira vrijednost $x$, izračunava $z_x=g^x mod p$ i
šalje primatelju (R).
\item Napadač zaprima vrijednost $z_x$. Potom generira svoju vrijednost $n$ i
šalje započinjatelju vrijednost $z_n$. Započinjatelj I i napadač na kraju imaju
dogovoren ključ
$K_{I'}={z_x}^n mod p$.
\item S druge strane napadač započinje dogovor ključa sa stranom R s
vrijednostima $n$ i $z_n$. Primatelj R i napadač na kraju imaju dogovoren ključ
$K_{R'}={z_y}^n mod p$.
\end{enumerate}
\nomenclature{MITM}{Napad čovjek u sredini}

Kao rezultat navedenog napada napadač ima dogovoren ključ $K_{I'}$ sa
započinjateljem i ključ $K_{R'}$ s primateljem i može primati šifrirani promet s
jedne strane, dešifrirati ga i ponovo šifrirati kako bi ga poslao drugoj strani.

Za zaštitu protokola DH od napada čovjek u sredini stvoreni su
novi protokoli koji štite DH razmjenu kako bi onemogućili napadača
da utječe na uspostavu zajedničke tajne. Protokol STS (engl.
\emph{Station-to-Station}) je prvi protokol koji sprječava napad
čovjek u sredini korištenjem autentifikacije. Autentifikacija se postiže
korištenjem asimetričnog algoritma za digitalno potpisivanje poruka u
komunikaciji. Za digitalni potpis poruka potrebno je razmijeniti certifikate s
kojima će se potpisivati. Certifikati se mogu razmijeniti u posebnoj inačici
protokola STS koja se zove potpuni STS (engl. \emph{Full STS}  \cite[str.
1256]{van2011encyclopedia}.


Razmjena poruka u potpunoj inačici protokola STS definirana je na sljedeći
način:
\begin{itemize}
\item Započinjatelj I generira tajnu vrijednost $x$, izračunava $z_x=g^x mod p$
i šalje primatelju R.
\item Primatelj R generira tajnu vrijednost $y$ i izračunava $z_y=g^y mod p$. Uz
to izračunava i ključ $K={z_x}^y mod p$ i s njim šifrira digitalno potpisane
podatke $z_y$ i $z_x$. Konačna poruka koja se šalje započinjatelju koja uključuje
certifikat $Cert_R$ je sljedeća:
$z_y, Cert_R, E_K(S_R(z_y,z_x))$
\item Započinjatelj I prima poruku i izračunava ključ $K$ s pomoću kojeg
dešifrira $S_R(z_y,z_x)$ i provjerava digitalni potpis korištenjem certifikata
$Cert_R$. Ako je provjera uspješna započinjatelj odgovara sa sličnom porukom
gdje je promijenjen redoslijed podataka digitalnom potpisu:
$Cert_I, E_K(S_I(z_x,z_y))$.
\end{itemize}
\no{
\begin{enumerate}
\item Započinjatelj (I) generira nasumičnu vrijednost $x$ i izračunava $z_x=g^x
    mod p$ i šalje prema primatelju (R).
\item Primatelj R generira nasumičnu vrijednost $y$ i na isti način računa
    $z_y=g^y mod p$. Potom izračunava tajni ključ $K_R={z_x}^y$ i digitalni potpis s
    pomoću svog privatnog ključa $SK_R$ od $z_x$ i $z_y$, $S_R=E_{SK_R}(z_x,z_y)$.
    Taj digitalni potpis potom šifrira korištenjem ključa $K_R$ i šalje
    započinjatelju $z_y, E_{K_R}(S_R)$.
\item Započinjatelj po primitku izračunava tajni ključ $K_I={z_y}^x$, dešifrira $S_R$
    jer su ključevi $K_I$ i $K_R$ jednaki. Potom s javnim ključem primatelja
    $PK_R$ provjerava valjanost digitalnog potpisa. Ako je provjera uspješna
    onda primatelju šalje šifrirani digitalni potpis koji se izračunava na
    istovjetan način kao i kod primatelja: $E_{K_I}(S_I)$ gdje je
    $S_I=E_{SK_I}(z_y,z_x)$.
\item Primatelj R dešifrira podatke pomoću prethodno izračunatog ključa $K_R$
    i pomoću javnog ključa $PK_I$ provjerava valjanost digitalnog potpisa. U
    slučaju da je provjera uspješna primatelj i započinjatelj su uspješno
    autentificirano dogovorili zajedničku tajnu.
\end{enumerate}
U slučaju da se započinjatelj i primatelj prethodno ne poznaju, u sklopu druge i
treće poruke
šalju se digitalni certifikati s javnim ključevima.
}

Uvođenjem digitalnog potpisa, odnosno korištenjem javnih ključeva, osigurano je
da napadač nije u
mogućnosti izravno ugroziti dogovor i utjecati na dogovorenu zajedničku tajnu.
Pritom postoje druga dva sigurnosna problema: mogućnost vezivanja krivog
identiteta za dogovorenu zajedničku tajnu (engl. \emph{identity misbinding
attack}) i problem korištenja dogovorene zajedničke tajne u sklopu dogovora što
ne zadovoljava zahtjev korištenja odvojenih ključeva za različite kriptografske
operacije (engl. \emph{key separation}) \cite{Krawczyk2003sigma}.

\subsection{SIGMA}
\label{sec:sigma}

\nomenclature{SIGMA}{protokol \emph{SIGn and MAc}}

Protokol SIGMA (engl. \emph{SIGn and MAc}) je protokol za autentificirano
dogovaranje zajedničke tajne koji je zasnovan na korištenju digitalnog potpisa i
MAC algoritma za zaštitu dogovora. Protokol SIGMA je formalno verificiran i
predstavlja rješenje koje omogućuje siguran dogovor zajedničke tajne
\cite{Krawczyk2003sigma}.

Dogovaranje zajedničke tajne u protokolu SIGMA odvija se na sljedeći
način:
\begin{enumerate}
\item Započinjatelj (I) generira nasumičnu vrijednost $x$ i izračunava $z_x=g^x
    mod p$ i šalje prema primatelju (R).
\item Primatelj (R) generira nasumičnu vrijednost $y$ i izračunava $z_y=g^y mod
    p$. Potom izračunava tajni ključ $K_R$ pomoću KDF (engl. \emph{key
    derivation function}, pojam je objašnjen u sljedećem poglavlju) funkcije za
    izračunavanje ključeva iz zajedničke tajne $SS_R=z_x^{y}$, $K_R=KDF(SS_R)$.
    Zatim šalje odgovor započinjatelju koji sadrži $z_y$, javni ključ $PK_R$,
    digitalni potpis $S_R=E_{PK_R}(z_x, z_y)$, i MAC od javnog ključa $PK_R$
    zaštićen s tajnim ključem $K_R$, $M_R=MAC_{K_R}(PK_R)$.
\item Započinjatelj po primitku poruke izračunava ključ $K_I$ na način analogan
    primatelju (iz vrijednosti $z_y$ i $x$). Zatim provjerava dobiveni digitalni
    potpis $S_R$ i MAC $M_R$. Ako provjera uspije primatelju šalje poruku sa
    svojim javnim ključem $PK_I$, digitalnim potpisom $S_I=E_{PK_I}(z_y, z_x)$ i
    MAC od javnog ključa $PK_R$ zaštićen s tajnim ključem $K_I$, 
    $M_I=MAC_{K_I}(PK_I)$.
\item Primatelj po primitku provjerava valjanost digitalnog potpisa $S_I$ i
    MAC-a $M_I$. Ako su vrijednosti valjane, uspješno je dogovorena
    zajednička tajna.
\end{enumerate}

Ključna prednost protokola SIGMA je zaštita identiteta koji se razmjenjuje s
pomoću MAC funkcije uz ključ dobiven iz Diffie-Hellman izračuna. Na taj se
načini izravno vezuje identitet (javni ključ) dogovaratelja uz dogovorenu
zajedničku tajnu. To izravno onemogućava napad čovjek u sredini jer treća
strana nije u mogućnosti utjecati na dogovorenu zajedničku tajnu.

\section{Pseudo nasumične funkcije}

Pseudo nasumična funkcija (PRF, engl. \emph{pseudo random function}) je
deterministička funkcija koja za određeni ključ i ulaz daje rezultat koji se ne
može razlikovati od nasumične funkcije ulaza.
\cite[str. 994]{van2011encyclopedia}. Idealna PRF ima za cilj postizanje
neprepoznatljivosti između rezultata funkcije i nasumičnih podataka. 
Primjena pseudo nasumičnih funkcija vrlo je
široka u kriptografiji. Koriste se za dobivanje sigurnih ključeva iz
zajedničkih tajni, kao zamjena za korištenje generatora slučajnih brojeva,
za dobivanje niza ključeva iz jednog izvornog ključa te za izračunavanje
funkcija MAC za provjeru integriteta podataka. Pseudo nasumične funkcije se
najčešće izvode korištenjem funkcija kriptografskog sažetka.
\nomenclature{PRF}{\emph{Pseudo Random Function} - pseudo nasumična funkcija}

\subsection{Funkcije izračunavanja ključeva}
Nakon dogovora jedne zajedničke tajne, komunicirajuće strane žele koristiti tu
tajnu za zaštitu komunikacije. Dogovorena tajna može imati različite duljine pa
nastaje problem kako na siguran način dobiti ključ određene duljine za
kriptografske algoritme koji će se koristiti. Uz to, da bi se ostvarili svi
potrebni sigurnosni zahtjevi potrebno je više različitih vrsta kriptografskih
algoritama. Za različite vrste algoritama nije preporučljivo koristiti iste
tajne, već je potrebno korištenjem nekog algoritma izračunati različite ključeve za
svaki kriptografski algoritam. Ta dva zahtjeva se ostvaruju korištenjem funkcije za
izračun ključeva (engl. \emph{Key Derivation Functions} - KDF). Te se funkcije
koriste i u računalnim sustavim za pohranjivanje lozinki korisnika, kako bi se
mogućim napadačima otežalo razotkrivanje lozinki.
\nomenclature{KDF}{\emph{Key Derivation Function} - funkcija za izračunavanje
ključeva}

KDF se izvodi korištenjem pseudo-nasumičnih funkcija koje se izvode određenim 
brojem ponavljanja funkcija kriptografskog sažetka nad tajnim podacima. Funkcija
kriptografskog sažetka se u pravilu koristi u HMAC  načinu rada za potrebe
KDF-a. Broj ponavljanja određuje se ovisno o primjeni: veći je za pohranu
lozinki u odnosu na izračunavanje tajnih ključeva potrebnih za komunikaciju.

Standardizirani i najčešće korišteni algoritmi za KDF su sljedeći:
\begin{itemize}
\item PBKDF i PBKDF2 (engl. \emph{Password-Based KDF}) \cite{chen2008recommendation}
    koriste se za pohranu autentifikacijskih
podataka iz korisničkih lozinki. Izvodi se veliki broj iteracija kako bi se
napadaču otežalo pogađanje početne lozinke iz tih podataka. Preporučeni broj
iteracija je barem 1000.
\item HKDF (engl. \emph{HMAC-based Extract-and-Expand KDF})
\cite{rfc5869} namijenjen je za izračun tajnih ključeva iz dogovorene zajedničke
tajne i ne koristi dodatne iteracije. Svaki oktet neovisno se generira kako bi
se povećala sigurnost izračunatog ključa.  HKDF omogućuje definiranje dodatnih
varijabli kako bi se povećala kvaliteta ključeva.  Veći broj iteracija u ovom
slučaju nije poželjan jer se izračunavanje ključeva radi za vrijeme
komunikacije. Preveliki broj iteracija bi uzrokovao usporavanje komunikacije
\cite{krawczyk2010cryptographic}.
\end{itemize}
% vim: spell spelllang=hr
