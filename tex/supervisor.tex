\thispagestyle{empty}
\section*{O mentoru}

Miljenko Mikuc doktorirao je 1997. godine na Sveučilištu u Zagrebu u području
elektrotehnike. Trenutno je izvanredni profesor na Fakultetu elektrotehnike i
računarstva, Zavodu za telekomunikacije na istom sveučilištu. Njegova područja
istraživanja uključuju dizajn logičkih sklopova, mrežne protokole, simulacije
računalnih mreža i računalnu sigurnost. Sudjelovao je u 7 znanstvenih projekata
pod pokroviteljstvom Ministarstva znanosti, obrazovanja i sporta Republike
Hrvatske. Bio je voditelj 2 projekta pod pokroviteljstvom istog ministarstva i
voditelj zajedničkih istraživačkih projekata sa ``The Boeing Company- IDS, LabNet
Analysis, Modelling Simulation and Experimentation'', ``International Computer
Science Institute'' i ``The FreeBSD Foundation'' iz SAD-a te Ericssonom Nikola
Tesla d.d. iz Zagreba. Trenutno je voditelj istraživačkog projekta: ``Prilagođen
IMUNES za Ericsson (E-IMUNES)'' u suradnji sa tvrtkom Ericsson Nikola Tesla d.d.
Objavio je više od 40 radova u časopisima i na konferencijama u području
komunikacijskih mreža, protokola, virtualizacije, formalnih metoda i sigurnosti.

Nositelj je dva predmeta na preddiplomskom studiju: ``Digitalna logika'' i ``Mrežno
programiranje'', dva predmeta na diplomskom studiju: ``Sigurnost u Internetu'' i
``Upravljanje mrežom i uslugama'', i dva predmeta na postdiplomskom doktorskom
studiju: ``Odabrana poglavlja komunikacijskih protokola'' i ``Formalizmi u
telekomunikacijama''.

Pod njegovim vodstvom diplomirala su 123 studenta po programu ETF-4 i FER-1, 28
studenata obranilo je završne radove na preddiplomskom studiju, a 25
studenata obranilo je diplomske radove na diplomskom studiju po programu FER-2.
Na poslijediplomskom studiju, pod njegovim vodstvom, 15 studenata je obranilo
magistarske radove. Jedna studentica obranila je svoju doktorsku disertaciju.
Trenutno je mentor troje studenata na doktorskom studiju.

\newpage
\thispagestyle{empty}
\section*{About the Supervisor}
Miljenko Mikuc received his PhD in Electrical Engineering from University of
Zagreb, Croatia, in 1997. He is currently Associate Professor at the Faculty of
Electrical Engineering and Computing, Department of Telecommunications within
the same university. His research interests include digital logic design,
network protocols, network simulation and security. He participated in 7
scientific projects financed by the Ministry of Science, Education and Sports of
the Republic of Croatia. He was a project leader of 2 projects of Applications
of Information Technology financed by the same Ministry and the project leader
of cooperation research projects with ``The Boeing Company – IDS, LabNet
Analysis, Modelling Simulation and Experimentation'', ``International Computer
Science Institute'' and ``The FreeBSD Foundation'' from USA and with Ericsson
Nikola Tesla d.d. from Zagreb. Currently he is a project leader of the research
project: ``Ericsson Customized IMUNES (E-IMUNES)'' in cooperation with Ericsson
Nikola Tesla d.d. company. He published more than 40 papers in journals and
conference proceedings in the area of communication networks, protocols,
virtualization, formal methods and security.

He is a lecturer in charge on two undergraduate study courses: ``Digital Logic''
and ``Network Programming'', two graduate study courses: ``Internet Security'' and
``Network and Service Management'', and two postgraduate doctoral study
courses: ``Selected topics in communication protocols'' and ``Formalisms in
telecommunications''.

Under his supervision 123 students graduated on graduate study program ETF-4 and
FER-1, 28 students defended their bachelor thesis on undergraduate (BSc) study
program FER-2, 25 students defended their master thesis. On postgraduate master
study program, under his supervision 15 students defended their master thesis.
One student defended her doctoral thesis. He is currently dissertation advisor
for 3 PhD students.
