\chapter{Zaključak}

U doktorskoj disertaciji dizajniran je i programski ostvaren protokol za
sigurno dogovaranje kriptografskih algoritama i ključeva, koji omogućuje
kriptografski prilagodljivu komunikaciju između dva komunicirajuća uređaja.
Kriptografski prilagodljiva komunikacija ključna je u ostvarivanju dugoročno
sigurne komunikacije, što je prikazano u pregledu područja istraživanja.
Model protokola formalno je
specificiran i verificiran u alatu za automatsku provjeru sigurnosti
u odnosu na prethodno postavljene uvjete sigurne komunikacije. Primjena
formalne verifikacije omogućila je dizajn protokola koji zadovoljava sve željene
uvjete sigurne komunikacije.

Dogovor algoritama i ključeva u protokolu ACAP može se izvoditi na način koji je
neovisan o sloju, aplikaciji i trenutnom operacijskom sustavu. 
Analizirana
je integracija protokola u sustav za sigurnu komunikaciju u okolini Interneta
stvari s osvrtom na prilagodljivost odabira algoritma koji uzima u obzir
trenutne
mogućnosti uređaja. Arhitektura protokola omogućuje učinkovit dogovor preduvjeta
sigurne komunikacije u modelu klijent poslužitelj i mrežama ravnopravnih čvorova
s niskim zahtjevima na mrežnu propusnost i računalnu snagu komunicirajućih
strana. Programsko ostvarenje protokola omogućuje korištenje protokola na svim
razinama protokolnog složaja TCP/IP te pokazuje robusnost rješenja i
ostvarene sigurnosne mehanizme.

% vim: spell spelllang=hr
