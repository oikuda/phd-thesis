\chapter{Uvod}

Sigurna komunikacija je vrlo važan dio današnjeg računalnog svijeta i Interneta
uopće. Ona se ostvaruje korištenjem kriptografije, odnosno kriptografskih
algoritama i ključeva, za ostvarivanje osnovnih sigurnosnih zahtjeva
tajnosti, cjelovitosti i dostupnosti. Za ostvarivanje sigurnosnih zahtjeva
koriste se različite kombinacije kriptografskih algoritama. Primjerice, za
računanje digitalnog potpisa potrebni su algoritam kriptografskog sažetka i
algoritam asimetričnog kriptosustava uz potrebne ključeve.

Trenutno postoji velika količina različitih algoritama iste namjene koji ostvaruju
iste sigurnosne zahtjeve. Neprestano se razvijaju novi algoritmi i otkrivaju
ranjivosti prethodno dizajniranih algoritama. Uz to, zbog povećanja računalne
moći zahtjevi na veličinu kriptografskih ključeva neprestano rastu. Stoga je
potrebno osigurati siguran dogovor kriptografskih algoritama i ključeva koji će
se moći dinamički prilagođavati trenutnim uvjetima i zahtjevima za sigurnost
komunikacije. Dinamički dogovor kriptografskih algoritama i ključeva definiran
je u okviru pojma kriptografski prilagodljive komunikacije.

Kriptografski prilagodljiva komunikacija predstavlja princip u kojem se
inačice kriptografskih algoritama ne ugrađuju izravno u aplikacije i sustave,
već se samo definira koja se vrsta algoritma želi koristiti. To omogućuje
naknadnu promjenu inačice algoritma bez promjene aplikacije, odnosno sustava.
Suvremeni protokoli za sigurnu komunikaciju, poput TLS-a, IPsec-a i SSH-a,
koriste princip kriptografski prilagodljive komunikacije. Osnovni nedostatak tih
protokola je njihova složenost koja značajno otežava jednostavno programsko
ostvarenje i
formalnu verifikaciju njihovog dizajna. S druge strane, ti protokoli se ne mogu
koristiti na uređajima s ograničenim računalnim sposobnostima koji su sastavni
dio novih okolina temeljenih na Internetu stvari. Dogovor algoritama i ključeva
u tim protokolima nije moguće koristiti neovisno o protokolu i primijeniti na
druge komunikacijske slojeve i primjene zbog velike razine integracije i
povezanosti s komunikacijskim slojem i protokolom.

Stoga je u doktorskoj disertaciji dizajniran i izveden protokol za sigurno
dogovaranje
kriptografski prilagodljive komunikacije koji je neovisan o komunikacijskom
sloju i može se koristiti na uređajima ograničenih računalnih sposobnosti.
Jednostavna arhitektura protokola omogućila je formalno modeliranje protokola i
računalnu formalnu verifikaciju tog modela. Dodatno, jednostavnost arhitekture
smanjuje vjerojatnost grešaka u programskom ostvarenju i omogućava lakše održavanje
sustava temeljenog na tom protokolu. Opisana je izravna primjena
modeliranog protokola u okruženju Interneta stvari koja se temelji na prilagodbi
odabira algoritma ovisno o trenutnim mogućnostima uređaja i potrebnoj razini
sigurnosti. Uz disertaciju je izvedeno i programsko ostvarenje koje je ispitano u
različitim komunikacijskim uvjetima u emuliranoj mrežnoj okolini u alatu
IMUNES.

U doktorskoj disertaciji ostvareni su sljedeći znanstveni doprinosi:
\begin{enumerate}
\item Protokol za sigurno dogovaranje kriptografskih algoritama neovisan o sloju
protokolnog složaja, aplikaciji i operacijskom sustavu. 
\item Metode za integraciju protokola u aplikacije koje koriste različite modele
komunikacije, a posebice mreže ravnopravnih čvorova i model klijent-poslužitelj.
\item Ocjena učinkovitosti protokola u simuliranom i stvarnom mrežnom okružju.
\end{enumerate}

Protokol je definiran u poglavlju \ref{ch:model} i formalno verificiran u
poglavlju \ref{ch:verif}. Metode za integraciju protokola u okolinu Interneta
stvari prikazane su u poglavlju \ref{ch:iot}, dok je arhitektura za integraciju
protokola prikazana na početku poglavlja \ref{ch:impl}. U drugom dijelu
poglavlja \ref{ch:impl} prikazana je ocjena učinkovitosti protokola na temelju
mjerenja performansi protokola u različitim mrežnim uvjetima.

U poglavlju \ref{ch:crypto} dan je pregled vrsta kriptografskih algoritama.
Kriptografski algoritmi su podijeljeni u tri osnovne skupine: algoritmi za
stvaranje kriptografskih sažetaka, simetrični kriptografski algoritmi i
asimetrični kriptografski algoritmi.
Kriptografski sažetak služi za zaštitu cjelovitosti podataka u
digitalnim potpisima ili zaštitnim sumama. Algoritmi simetričnog
kriptosustava služe osiguravanje tajnosti podataka, a algoritmi asimetričnog
kriptosustava se koriste za dogovaranje i prijenos ključeva te digitalno
potpisivanje podataka. Na kraju poglavlja obuhvaćeni su asimetrični algoritmi za sigurno
dogovaranje zajedničke tajne koji su temelj protokola opisanog u disertaciji.

Poglavlje \ref{ch:agility} sadrži opis trenutno korištenih protokola za sigurnu
komunikaciju koji koriste principe prilagodljive komunikacije, na kojima je
temeljen predloženi protokol za sigurno dogovaranje kriptografski prilagodljive
komunikacije (engl. \emph{Agile Cryptographic Agreement Protocol} - ACAP).
Opisani su dijelovi tih protokola i objašnjena je njihova
interna arhitektura. Na kraju poglavlja opisani su novi protokoli za sigurnu
komunikaciju koji se temelje na jednostavnijoj arhitekturi za postizanje
sigurne komunikacije u različitim okruženjima.

\nomenclature{ACAP}{\emph{Agile Cryptographic Agreement Protocol}}

Poglavlje \ref{ch:model} sadrži opis protokola ACAP koji služi za siguran
dogovor kriptografskih algoritama i ključeva za osiguravanje buduće
komunikacije. Prikazane su sve poruke u protokolu i opisana je njihova razmjena
uz pregled kriptografskih operacija potrebnih za siguran dogovor. Analiziran je
algoritam dogovora kriptografskih algoritama uz pregled sigurnosnih mehanizama
ugrađenih u protokol.

U poglavlju \ref{ch:verif} opisan je jezik SPDL (engl. \emph{Security Protocol
Definition Language}) koji je korišten za
definiranje i verifikaciju formalnog modela protokola ACAP na jednostavnom
primjeru. Potom je opisan model protokola i sigurnosnih zahtjeva koji se
automatski verificiraju korištenjem alata Scyther. Na kraju poglavlja prikazani su
rezultati verifikacije uz analizu zahtjevnosti postupka verifikacije.

Primjena protokola ACAP u okolini Interneta stvari prikazana je u poglavlju
\ref{ch:iot}. Na početku poglavlja opisana je arhitektura okoline Interneta
stvari i prepoznate su ključne primjene protokola na komunikaciju u toj
okolini. Potom je dan formalan model sigurne komunikacije koji predstavlja
minimalan skup objekata i operacija potrebnih za sigurnu komunikaciju, a
temeljen na dogovoru algoritama i ključeva kroz protokol ACAP. Na kraju
poglavlja prikazana je prilagodljiva procedura za odabir algoritama koja uzima
u obzir trenutne mogućnosti uređaja i željenu razinu sigurnosti za usporedbu
algoritama.

Poglavlje \ref{ch:impl} opisuje mogućnosti programskog ostvarenja
dizajniranog protokola ACAP na različitim komunikacijskim slojevima
koji uključuju protokole TCP, UDP, IP i Ethernet. Potom je dan pregled mjerenja
učinkovitosti protokola u sklopu emulirane mrežne okoline alata IMUNES. Mjerenja
demonstriraju nisku razinu zahtjevnosti protokola na mrežnu propusnost i
analiziraju složenost kriptografskih operacija uključenih u dogovor. Na kraju je
dan pregled ostvarenih sigurnosnih mehanizama koji su definirani u opisu
protokola ACAP.

% vim: spell spelllang=hr
