\thispagestyle{empty}
\section*{Sažetak}

Sigurna komunikacija ključan je dio suvremenih sustava za komunikaciju i može se
postići korištenjem kriptografskih algoritama i protokola.
Međutim, i u takvom sustavu sigurnost komunikacije može biti ugrožena ako se
koriste
kriptografski algoritmi kojima su otkrivene ranjivosti. Koncept kriptografski
prilagodljive komunikacije omogućuje dinamičku promjenu kriptografskih
algoritama i ključeva za vrijeme rada sustava bez promjene izvedbene logike tog
sustava.
U doktorskoj disertaciji dizajniran je protokol za
sigurno dogovaranje kriptografski prilagodljive komunikacije koji, za razliku od
postojećih rješenja, može neovisno o komunikacijskom sloju i korištenoj
platformi dogovoriti preduvjete za daljnju sigurnu komunikaciju. Model protokola
je formalno verificiran u alatu za automatsku verifikaciju sigurnih postavki
protokola. Dogovor preduvjeta izveden je na učinkovit način koji ne postavlja
visoke zahtjeve na računalnu moć i propusnost mrežne komunikacije. Izvedena je
integracija protokola za dogovor u okolini Interneta stvari uzimajući u obzir
mogućnosti različitih uređaja u okolini. Dodatno, izvedeno je programsko
ostvarenje koje
omogućava korištenje protokola u raznim mrežnim modelima i okruženjima na
različitim mrežnim slojevima. U sklopu emulirane mrežne okoline pokazana je
učinkovitost mrežne komunikacije na različitim komunikacijskim slojevima i
korištenjem različitih vrsta kriptografije uz osvrt na ugrađene sigurnosne
mehanizme.

U okviru doktorske disertacije ostvareni su sljedeći znanstveni doprinosi:

1. Protokol za sigurno dogovaranje kriptografskih algoritama neovisan o sloju
protokolnog složaja, aplikaciji i operacijskom sustavu.

2. Metode za integraciju protokola u aplikacije koje koriste različite modele
komunikacije, a posebice mreže ravnopravnih čvorova i model klijent-poslužitelj.

3. Ocjena učinkovitosti protokola u simuliranom i stvarnom mrežnom okružju.

\vspace{1cm}
\textbf{Ključne riječi}: kriptografska prilagodljivost, sigurna komunikacija,
autentificirani dogovor ključeva, dogovor algoritama, neovisnost o sloju,
formalno verificirani model

\no{Kako se ne bi
ugrožavala sigurnost komunikacije potrebno je koristiti koncept kriptografski
prilagodljive komunikacije koji omogućuju zamjenu kriptografskih algoritama
kojima su otkrivene ranjivosti i promjenu ključeva za zaštitu komunikacije.
Koncept kriptografski prilagodljive komunikacije omogućuje dinamičku promjenu
kriptografskih algoritama i ključeva za vrijeme rada sustava bez promjene
izvedbene logike tog sustava.}
