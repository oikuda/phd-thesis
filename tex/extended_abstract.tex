\thispagestyle{empty}
\section*{Extended abstract}

{\Large{\textbf{Secure layer-agnostic agreement protocol for cryptographically
agile communication}}} \\

Securing data and communication channels implies the use of cryptography. To
achieve primary security requirements (confidentiality, integrity and
availability) different types of cryptographic algorithms are used.
Cryptographic algorithms can be divided into three groups: symmetric encryption,
asymmetric encryption and cryptographic hashes. These algorithms are often used
in different combinations to achieve better performance and various security
requirements.
There is a vast amount of cryptographic algorithms currently available. They are
all constantly being improved but at the same time new vulnerabilities are being
discovered and exposed. An algorithm with vulnerabilities, from the
cryptanalysis point of view, can still be safe for usage, but the identified
vulnerability points to a weakness in its definition. Such a
weakness can eventually be used for various attacks on systems that are secured
by that algorithm, so vulnerable algorithms should be replaced as soon as
possible.

However, replacing cryptographic algorithms that are tightly integrated into
communication protocols would eventually require changing the protocols
themselves. This usually means that the running software or firmware needs to be
updated. Changing protocols may prove to be an issue when
dealing with typically large and complex communication systems. This is the
reason why cryptographic algorithms should never be tightly integrated into a
communication protocol and the solution is to use the concept of cryptographic
agility. Cryptographic agility is characterized by the ability to change
cryptographic algorithms and keys while using the same protocols and deployed
systems. On the other hand, using agility makes it possible to use the same
cryptographic algorithms or protocols over a wide array of communication
protocols thus enabling easier replacement of potentially insecure cryptographic
algorithms.

Cryptographic agility is a known principle that is already integrated into most
widely deployed secure channel protocols, such as IPsec, SSL, TLS and SSH.
However, each of the implemented agility mechanisms is strictly tied to that
solution and cannot be easily extracted as a standalone mechanism
applicable for other purposes. The possible solution that would enable this
functionality is an adaptable cryptographically agile protocol that is proposed
in this thesis. This solution can be used on various network layers,
architectures and devices. Furthermore, the aim is the agreement of
cryptographic prerequisites (algorithms and keys) in order to enable various
security requirements in an interconnected environment. Using the proposed
cryptographically agile solution enables protection of protocols that don’t have
any security mechanisms deployed and enable development of new secure protocols
on top of the agreed primitives.

The doctoral thesis describes a novel, lightweight and adaptable solution which
is
easily applicable in existing networked environments and currently emerging
areas, such as Internet of Things (IoT).
Security isn’t always integrated in current IoT solutions and is typically
planned at a later stage as an upgrade. The communication between IoT devices
needs to be secure which requires an agreement protocol to enable communicating
parties to agree on a cryptographic algorithm and keys used to protect the
exchanged messages. The IoT ecosystem needs a flexible and adaptable agreement
mechanisms because IoT devices have limited computing and bandwidth resources.
Secure communication is possible only after communicating devices agree on a set
of cryptographic algorithms and keys. Since IoT services are driven by the
underlying data sources, it is critical to validate a sensor as a credible data
source and to protect the exchanged data.

\subsection*{Chapter 2 - Cryptography}

The second chapter describes basic cryptographic primitives, cryptographic
hashes, symmetric and asymmetric cryptosystems and focuses on their application
in current communication systems. It specifically describes the primitives used
for achieving an agile and adaptive agreement on cryptographic keys and
algorithms. The needed primitives include a HMAC (Hashed Message Authentication
Code) algorithm based on cryptographic hashes with a secret key, digital
signatures for authentication that combine cryptographic hashes with asymmetric
encryption algorithms and the Diffie-Hellman key exchange that uses asymmetric
cryptography to generate secret data to conduct a secure exchange. These three
mechanisms are needed to conduct a SIGMA (sign-and-mac) exchange between two
communicating parties. The SIGMA exchange represents an authenticated key
exchange algorithm that secures the exchange from man in the middle attacks by
protecting the data used for digital signature verification with HMAC and
vice versa. The resulting secret data needs to be further processed by
using a key derivation function (KDF) based on hash functions (Hashed KDF). This
is to ensure that keys generated from the same secret data are computationally
independent. Computational independence represents that the discovery of one key
won't reveal any data on other keys calculated from the same secret data.

\subsection*{Chapter 3 - Cryptographycally agile communication}

The third chapter describes modern cryptographically agile protocols. Most of
these protocols are secure channel protocols that have many components and thus
tend to be fairly complex (e.g. SSL/TLS, IPsec, SSH). That complexity makes it
hard, or even impossible in certain cases, to create a formal model that can be
thoroughly verified to achieve provable security. The lack of security proofs is
probably one of the causes of the recent increase of attacks on SSL/TLS.
Verification of certain solutions can also reveal previously unknown
vulnerabilities. Protocol complexity also increases the possibility
of implementation problems which can cause significant security problems even if
the protocol design is provably secure. The chapter focuses on describing
established cryptographically agile solutions and compares them to modern
solutions like tcpcrypt, MinimaLT and QUIC that employ a less complex approach
to secure channels. All these solutions are compared to ACAP, the lightweight
layer-agnostic agreement protocol described in the thesis. The main advantages
of ACAP are a layer-agnostic architecture that enables its application on all
communication layers and a simple architecture that enables formal verification
and reduces the possibility of implementation problems. ACAP represents an easy
to integrate and adaptable building block that distributes secure communication
prerequisites i.e. cryptographic keys and algorithms.

\subsection*{Chapter 4 - Secure cryptographic agreement protocol}

The ACAP protocol is described in the fourth chapter. It consists of four
messages: \initi{}, \initr{}, \listi{} and \listr{}. After the message
exchange,
both communicating parties have access to the following data: secret key
generated from the Diffie-Hellman shared secret, public keys of both sides for
authentication and a list of cryptographic algorithms supported by both sides.
The list of supported cryptographic protocols is divided into groups and agreed
upon for each group. The first algorithm that is on the client list and is
present on the server list is chosen as the agreed algorithm. ACAP employs the
following security mechanisms: computational independence of secret keys by
using HKDF, denial of service attack mitigation by separating the most expensive
operations in an asynchronous background process, trust establishment that
enables both manual and automatic trust verification through PKI. Additionally,
the SIGMA design prevents the man in the middle and replay attack on the
protocol. At the end of the chapter a comprehensive comparison of the SSH
transport layer protocol negotiation and ACAP is given.

\subsection*{Chapter 5 - Formal model specification}

The formal security model and formal model verification is described in chapter
five. The formal security model is described in the security protocol
description language (SPDL) used by the Scyther security protocol verifier.
Models in SPDL are defined by using protocols and roles. Each protocol can
contain two or more roles that define multiple send and receive events. A
comprehensive explanation of the model specification process is given on the
Needham-Schroeder-Loewe protocol is give. After that the ACAP protocol model is
described. The model is specified by using two protocols, a main protocol with
the initiator and responder roles and a helper protocol that enabled the
specification of the Diffie-Hellman key exchange. After the message exchange
specification the following security requirements (Scyther claims) were defined,
immutability of the DH exponentials, secrecy of the shared secret used to derive
session keys and the inability of the attacker to interfere with the message
exchange. All three security requirements were successfully verified by using an
unbounded amount of protocol runs.

\no{
The main design motivation was to achieve provable security by extracting and
integrating
key exchange with algorithm agreement into a standalone solution. As a result,
the designed lightweight solution is formally verified and achieves
distribution of keys and algorithms in a provably secure manner. The conducted
formal verification proves resilience to (pre)replay attacks and
man-in-the-middle attacks.
}

\subsection*{Chapter 6 - Secure communication in the Internet of things}

Chapter six describes how the designed solution can be introduced in the
Internet of Things environment. It focuses on the device capabilities and
introduces three device tiers, sensors, middle and cloud tier. All secure
communication scenarios are conducted between devices from all tiers. The main
proposed secure communication scenarios include authenticated device management,
reliable data collection and sensitive data protection that are built on a
platform independent architecture. After scenario descriptions a secure
communication model for the IoT environment is given. The model consists of
devices, public keys, communication layers, hardware capabilities, cryptographic
algorithms and a set of secure communication operations. Secure communication
operations are performed to enable all previously described communication
scenarios. Furthermore, this chapter describes an adaptive cryptographic
algorithm agreement procedure that takes into account the required level of
security, cryptographic algorithm complexity, device battery power, bandwidth
and processing power to choose a cryptographic algorithm that is best suited in
the current environment.

\subsection*{Chapter 7 - ACAP protocol implementation}

A prototype implementation in Python and measurement results are described in
chapter seven. 
The proposed prototype architecture is composed from five key components:
calculation and verification for cryptographic operations, message handling,
layer independent communication, cryptographic algorithm agreement and
communication with external applications. The prototype implementation supports
communication on the following layers: Ethernet, IPv4, IPv6, TCP, UDP and SCTP
with the same message formats that are shown in the appendix. It also supports
standard and elliptic curve asymmetric algorithms for Diffie-Hellman exchanges
and digital signatures.The prototype was tested in an emulated network
environment in the IMUNES tool, which simplified network setup and enabled
reliable measurements through environment automation. The measurements include
the effect of public key size on protocol message size, agreement duration in
regards to link delay and available bandwidth. They have shown that the solution
is bandwidth efficient as the entire negotiation is performed in under two
seconds on a 10 Kbps link. The use of elliptic curve cryptography is encouraged,
especially for devices with constrained resources. Furthermore, this chapter
provides an overview of cryptographic operation complexity for client and
server which shows the deployed denial of service mitigation mechanisms.

Finally, chapter eight concludes the thesis by giving a comprehensive overview
of the designed layer-agnostic protocol that enables cryptographic agility for
various devices and platforms. 

The scope of this doctoral thesis covers the following scientific contributions:

1. Secure layer-agnostic agreement protocol for cryptographically agile
communication that is application and operating system independent.

2. Methods for integrating the protocol with applications that use different
communication models, especially P2P networks and client-server environments.

3. Evaluation of protocol communication and agreement efficiency in a simulated
and real network environment.

\vspace{1cm}
\textbf{Keywords}: cryptographic agility, secure communication, layer-agnostic
design, authenticated key exchange, algorithm agreement, formal model
verification

\no{
Secure communication is an important aspect of today’s interconnected
environments and it can be achieved by the use of cryptographic algorithms and
protocols. However, many existing cryptographic mechanisms are tightly
integrated into communication protocols. Issues emerge when security
vulnerabilities are discovered in cryptographic mechanisms because their
replacement would eventually require replacing deployed protocols. The concept
of cryptographic agility is the solution to these issues because it allows
dynamic switching of cryptographic algorithms and keys prior to and during the
communication. Most of today’s secure protocols implement cryptographic agility,
but cryptographic agility mechanisms cannot be used in a
standalone manner. In order to deal with the aforementioned limitations, this
thesis describes a lightweight cryptographically agile agreement protocol. The
design of the protocol is formally verified in an automated security
verification tool. The agreement of cryptographic keys and algorithms is done in
a computationally efficient way and has a low communication bandwidth
requirement. Protocol integration and usage is demonstrated in an Internet of
things environment with special emphasis on the agreement procedure which adapts
to current device capabilities. Additionally, a prototype implementation which
can communicate on different network layers is presented. The implementation is
tested and evaluated in an emulated network environment. The tests demonstrate
agreement efficiency on different communication layers when using different
types of cryptography.
}
