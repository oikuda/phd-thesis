\chapter{Protokol za sigurno dogovaranje kriptografskih algoritama}
\label{ch:model}

U ovom poglavlju opisan je novi protokol za sigurno dogovaranje
kriptografskih algoritama neovisan o komunikacijskom sloju, u daljnjem tekstu
ACAP (engl. \emph{Agile Cryptographic Agreement Protocol}). 
Osnovni zahtjevi na dizajn protokola ACAP su sljedeći:
\begin{itemize}
    \item sigurna razmjena kriptografskih ključeva,
    \item prilagodljivi dogovor kriptografskih algoritama,
    \item jednostavna arhitektura koja omogućava formalnu verifikaciju,
    \item neovisnost o komunikacijskom sloju i uređaju i
    \item učinkovita komunikacija s malim brojem poruka.
\end{itemize}

Protokol služi za komunikaciju između dvije strane (klijenta i poslužitelja) i
omogućava sigurno dogovaranje svih potrebnih preduvjeta kako bi se ostvarila buduća
sigurna komunikacija između te dvije strane gdje je klijent strana koja
započinje komunikaciju, a poslužitelj druga strana koja očekuje poruku za
početak dogovora. Protokol ACAP omogućuje i dogovor u sklopu paradigme
ravnopravnih čvorova gdje su svi čvorovi istovremeno klijenti i poslužitelji.

Nakon uspješne razmjene poruka obje strane imaju pristup sljedećim
podacima:
\begin{itemize}
    \item Tajnom ključu koji se generira iz zajedničke tajne koja se uspostavlja
	korištenjem protokola Diffie-Hellman \cite{DH}
	i omogućava uspostavljanje principa PFS (engl. \emph{Perfect Forward
	Secrecy}) \cite{krawczyk2011perfect}. Princip PFS omogućava tajnost
	razmjene podataka i u slučaju da se otkrije dugoročna zajednička tajna.
    \item Javnim ključevima koji se koriste za autentifikaciju komunicirajućih
	strana.
    \item Popisu kriptografskih algoritama koji su podržani na obje
	komunicirajuće strane te koji će se koristiti za ostvarivanje potrebnih
	sigurnosnih zahtjeva.
\end{itemize}

Na početku potpoglavlja \ref{sec:oznac} dan je pregled označavanja za sve
kriptografske operacije
koje se koriste za opis protokola. Potom slijedi opis poruka i razmjene poruka s
kojima se ostvaruje komunikacija u protokolu ACAP. Na kraju poglavlja
analizira se sigurnost predloženog protokola i opisuju se sigurnosni mehanizmi
koji su integrirani u protokol.

Protokol ACAP temeljen je na principu \emph{sign-and-mac} (SIGMA) koji je
opisan u poglavlju \ref{sec:sigma}. SIGMA kombinira razmjenu
ključa korištenjem Diffie-Hellman algoritma, digitalni potpis i algoritam HMAC
za zaštitu dogovora tako da
onemogućuje napadača da uspostavi napad čovjek u sredini. Protokol JFK, koji
je opisan u potpoglavlju \ref{sec:jfk}, također koristi princip SIGMA, što
objašnjava sličnost prve dvije poruke protokola ACAP s ovim protokolom. Dogovor
algoritama koji će se koristiti za osiguravanje komunikacije sličan je onom
koji je definiran u transportnom dijelu protokola SSH.

\section{Označavanje kriptografskih operacija}
\label{sec:oznac}

Za opis protokola koristi se standardni način označavanja kriptografskih
operacija. Oznaka $X$
predstavlja jednu stranu u komunikaciji, a oznaka $x$ predstavlja tajnu
komponentu Diffie-Hellman razmjene na odgovarajućoj strani. Klijent se označava
oznakom $I$ (od
engl. \emph{initiator}, započinjatelj), a poslužitelj oznakom $R$ (od
engl. \emph{responder}, odgovaratelj):

{
\renewcommand{\arraystretch}{1.40}
\begin{tabular}{p{1.0cm} p{13.5cm}}
$g^x	$ & javni dio Diffie-Hellman (DH) razmjene ključeva, uz specifikaciju DH
	    grupe koja se želi koristiti za komunikaciju, \\

$N_X	$ & jedinstvena privremena vrijednost (engl. \emph{nonce}) koja se koristi za
	    jedan dogovor, \\

$K	$ & ključ sjednice, rezultat je KDF operacije nad dijeljenom
	    tajnom (dijeljena tajna se računa iz $(g^x, y)$ ili $(g^y, x)$,
	    ovisno o komunikacijskoj strani, gdje su $x$ i $y$ tajni dijelovi
	    DH razmjene) i \emph{nonce} vrijednosti obje strane,

	    $K = KDF (g^{xy},N_I,N_R) $, \\

$PK_X	$ & javni ključ (ili digitalni certifikat), \\

$S_X\{\}$ & digitalni potpis s tajnim (privatnim) ključem koji se može
	    provjeriti s odgovarajućim javnim ključem $PK_X$,\\

$H_K\{\}$ & HMAC operacija uz korištenje ključa sjednice K, \\

$L_X	$ & popis podržanih kriptografskih algoritama, podijeljen u kategorije, \\

$L_{IR} $ & dogovoreni kriptografski algoritmi između klijenta ($I$) i poslužitelja ($R$)
	    tijekom ACAP razmjene, \\

$K_{IR} $ & tajni ključ koji je dogovoren između klijenta ($I$) i poslužitelja
	    ($R$), \\

$E_X	$ & poruka koja sadrži podatke o grešci koja se dogodila tijekom
	    komunikacije (koristi se samo u porukama ABORT). \\
\end{tabular}
}

\section{Poruke i tok komunikacije}
\label{sec:poruke}

Standardni tok komunikacije u protokolu ACAP prikazan je na
slici~\ref{msc}. U njemu je prikazano računanje i distribucija svih parametara
koji su potrebni za sigurnu komunikaciju.

\begin{figure}[ht]
\setlength{\instdist}{2.5cm}
\begin{centering}
\begin{msc}[c]{Protokol ACAP}
\declinst{c}{}{Klijent (I)}
\declinst{s}{}{Poslužitelj (R)}

\action*{generiranje $g^i,N_I$}{c}
\nextlevel[2]

\mess{\initi}{c}{s}
\mscmark[br]{$g^i,N_I$}{s}
\nextlevel[2]
\action*{generiranje $g^r,N_R$}{s}
\nextlevel[2]

\action*{$K=KDF(g^{ir},N_I,N_R)$}{s}
\nextlevel[3]

\mess{\initr}{s}{c}
\mscmark[bl]{$g^r,N_R,PK_R$}{c}
\nextlevel[2]

\action*{$K=KDF(g^{ri},N_I,N_R)$}{c}
\nextlevel[3]

\mess{\listi}{c}{s}
\mscmark[br]{$PK_I,L_I$ $$}{s}
\nextlevel[2]

\mess{\listr}{s}{c}
\mscmark[bl]{$L_R$}{c}
\nextlevel[2]

\referencestart{p}{\\$L_{IR},PK_I,PK_R,K_{IR}$}{c}{s}
\nextlevel[3]
\referenceend{p}

\end{msc}
\caption{Grafički prikaz razmjene poruka protokola ACAP}
\label{msc}
\end{centering}
\end{figure}


Protokol ACAP sastoji se od četiri osnovne poruke s pomoću kojih se izmjenjuju
javni ključevi, dogovara zajednička tajna te kriptografski algoritmi koji će se
koristiti za zaštitu daljnje komunikacije. 
Oznake u zagradama korištene u opisu poruka protokola označuju smjer poruke ($I
\rightarrow R$ označuje poruku koju klijent $I$ šalje poslužitelju $R$):

\begin{tabular}{p{3.0cm} p{8.0cm}}
    \initi($I \rightarrow R$): & $g^i, N_I$  \\
    \initr($R \rightarrow I$): & $g^r, N_R, PK_R, S_R\{g^r\}$,
    $H_K\{PK_R, N_I, N_R\}$ \\
    \listi($I \rightarrow R$): & $PK_I, L_I, S_I\{L_I,g^i,g^r\}$,
    $H_K\{PK_I, N_R, N_I\}$ \\
    \listr($R \rightarrow I$): & $L_R, S_R\{L_R,g^i,g^r,N_I,N_R\}$ \\
\end{tabular}
\vspace{5pt}

Klijent pokreće komunikaciju slanjem poruke \initi{}. Ona sadrži 
javni dio Diffie-Hellman razmjene klijenta ($g^i$) uz specifikaciju DH
grupe i klijentov
\emph{nonce} $N_I$. Ako poslužitelj podržava predloženu grupu odgovara
klijentu s porukom \initr{}.

Poruka \initr{} sadrži javni dio DH razmjene poslužitelja $g^r$, javni ključ
$PK_R$ poslužitelja i \emph{nonce} $N_R$. Poruka je zaštićena s pomoću digitalnog
potpisa javnog dijela DH razmjene $S_R\{g^r\}$ i HMAC-a javnog ključa i obje
privremene jedinstvene vrijednosti s ključem sjednice $K$ koji je uspostavljen
kroz DH, $H_K\{PK_R, N_I, N_R\}$.

\listi{} sadrži klijentov javni ključ $PK_I$ i popis podržanih
algoritama $L_I$. Digitalni potpis u $S_I\{L_I,g^i,g^r\}$ štiti popis podržanih
algoritama dok HMAC, koristeći ključ sjednice $K$, štiti klijentov javni ključ
$H_K\{PK_I, N_R, N_I\}$. Zadnja poruka u razmjeni je \listr{}. U toj
poruci prenosi se lista podržanih protokola poslužitelja $L_R$, koja je
zaštićena s digitalnim potpisom $S_R\{g^i,g^r,N_I,N_R,L_R\}$.

Nakon razmjene, klijent i poslužitelj imaju pristup popisu dogovorenih
algoritama $L_{IR}$, međusobnim javnim ključevima i zajedničkoj tajni $K_{IR}$
(poznatoj samo klijentu i poslužitelju).

\subsection{Rukovanje iznimkama}

Tijekom komunikacije moguće su iznimke koje će prouzročiti prekid izmjena
poruka. To je moguće u sljedećim slučajevima:
\begin{itemize}
\item različite Diffie-Hellman grupe u početnoj razmjeni,
\item pogrešan zaštitni sažetak HMAC,
\item pogrešan digitalni potpis ili
\item nepostojanje zajedničkih algoritama za korištenje kod klijenta i
poslužitelja.
\end{itemize}

U svrhu rukovanja iznimkama definiramo poruke ABORT koje su različite za
klijenta i poslužitelja:

\begin{tabular}{p{3.4cm} p{8.0cm}}
    \abortr($R \rightarrow I$): & $E_R, N_I, PK_R, S_R\{N_I, E_R\}$ \\
    \aborti($I \rightarrow R$): & $E_I, N_R, PK_I, S_I\{N_R, E_I\}$ \\
\end{tabular}
\vspace{5pt}

ABORT poruke sadrže kratak opis greške, odgovarajući \emph{nonce}, javni
ključ te digitalni potpis \emph{noncea} i opisa greške.

\section{Kriptografski algoritmi i ključevi za dogovaranje}

Kako bi komunicirajuće strane mogle na siguran način dogovoriti kriptografske
algoritme i ključeve potrebne za zaštitu komunikacije, obje strane u
komunikaciji moraju podržavati nekoliko osnovnih vrsta kriptografskih algoritama
za provedbu dogovora:
\begin{itemize}
    \item postupak razmjene ključeva Diffie-Hellman za dobivanje zajedničke
	tajne $g^{ir}$,
    \item operacija KDF za izračunavanje ključa sjednice $K$ i tajne za zaštitu
	naknadne komunikacije $K_{IR}$ iz zajedničke tajne dobivene DH postupkom
	i \emph{nonceva} $N_I$ i $N_R$,
    \item funkcija kriptografskog sažetka za izračun HMAC operacija s
	ključem sjednice $K$ i
    \item asimetrični algoritam za izračun i provjeru digitalnih potpisa tijekom
	dogovora.
\end{itemize}

Da bi se ostvario dogovor više različitih vrsta asimetričnih algoritama za zaštitu
komunikacije nužno je razmijeniti javne ključeve za sve podržane
asimetrične algoritme i na kraju odabrati ključ koji odgovara dogovorenom
asimetričnom algoritmu.

Za zaštitu komunikacije u prototipu protokola koriste se sljedeći algoritmi:
Diffie-Hellman 1024 MODP grupa \cite{rfc5114}, KDF operacija zasnovana na HMAC
inačici SHA-256 funkcije kriptografskog sažetka (HKDF \cite{rfc5869}), SHA-256
kriptografski sažetak za HMAC operacije i RSA-1536 kao asimetrični algoritam za
digitalne potpise. Uz te algoritme može se koristiti Diffie-Hellman
algoritam zasnovan na eliptičnim krivuljama, kao i ECDSA algoritam za digitalni
potpis (oba algoritma koriste ključeve veličine 256 bita).

\section{Dogovor kriptografskih algoritama}
\label{sec:dogovor}

Popisi podržanih kriptografskih algoritma podijeljeni su u kategorije. Te
kategorije se mogu definirati prema trenutnim potrebama zaštite. Za zapis
popisa koristi se format JSON~\cite{rfc4627}.
Na slici~\ref{fig:alg_list} vidi se primjer s više različitih kategorija i
skupova kriptografskih algoritama.
\begin{figure}[htb]
\begin{small}
\begin{verbatim}
{
    "hash":["SHA-256", "RIPEMD", "SHA-1", "MD5"],
    "secret_key":["AES-CTR_256", "AES-CBC_128", "3DES_192"],
    "public_key":["ECDSA_192", "DSA_2048", "RSA_2048"],
    "cryptographic_suite":[
      "TLS_ECDH_ECDSA_WITH_AES_128_GCM_SHA256",
      "TLS_DHE_PSK_WITH_AES_128_CBC_SHA256",
      "TLS_RSA_WITH_RC4_128_SHA" ]
}
\end{verbatim}
\end{small}
\vspace{-15pt}
\caption{Popisi kriptografskih algoritama i skupova koji se dogovaraju}
\label{fig:alg_list}
\end{figure}


Za dogovor kriptografskih algoritama u programskom ostvarenju koriste se principi iz
protokola SSH \cite{rfc4253}. Ovisno o
namjeni i okruženju u kojem se koristi protokol, moguće je definirati drukčije
algoritme dogovora koji će uzimati dodatne parametre prilikom odluke.
Prilagodba algoritma za dogovor za korištenje u okolini Interneta Stvari
prikazana je u poglavlju \ref{ch:iot}.

Kriptografski algoritmi odabiru se zasebno za svaku vrstu
algoritma: sažetak (\texttt{hash}), kriptografija tajnog ključa
(\texttt{secret\_key}), kriptografija javnog ključa (\texttt{public\_key}) ili
korištenjem skupa kriptografskih algoritama (\texttt{cryptographic suite}). Uz
to, moguć je i dogovor drugih parametara sigurne komunikacije, kao što je način
generiranja tajnih ključeva iz dobivene zajedničke tajne.
U popisu algoritama za kriptografiju javnog i tajnog ključa zadnji dio u imenu
algoritma predstavlja duljinu ključa koji je podržan, npr.
\texttt{AES-CTR\_256} predstavlja algoritam AES koji koristi CTR način
ulančavanja blokova s tajnim ključem duljine 256 bita.

Odabire se prvi algoritam s popisa klijenta koji se ujedno nalazi na popisu
poslužitelja. Primjer dogovora algoritama između klijenta i poslužitelja
prikazan je slikom \ref{fig:neg_lists}. Preduvjet je da svi algoritmi budu
poredani prema sigurnosti algoritma i duljini ključa. Algoritmi mogu biti poredani
i prema zahtjevnosti izračunavanja ukoliko se radi o okruženju s ograničenim
računalnim resursima. Prošireni model dogovora kriptografskih algoritama
za okolinu Interneta stvari prikazan je u poglavlju \ref{sec:iotprilag}.

\begin{figure}
\begin{subfigure}{0.33\linewidth}
\begin{framed}
\begin{small}
\begin{verbatim}
"hash":[
    "SHA-256",
    "RIPEMD",
    "SHA-1"],
"secret_key":[
    "AES-CTR_256",
    "AES-CBC_128",
    "3DES_192"],
"public_key":[
    "RSA_1024",
    "RSA_2048",
    "ECDSA_192"]
\end{verbatim}
\vspace{-15pt}
\end{small}
\end{framed}
\vspace{-10pt}
\caption{Popis algoritama klijenta}
\end{subfigure}
\begin{subfigure}{0.32\linewidth}
\begin{framed}
\begin{small}
\begin{verbatim}
"hash":[
    "SHA3-512",
    "SHA-512",
    "SHA-256"],
"secret_key":[
    "AES-CTR_256",
    "Salsa20_256",
    "AES-CBC_128"],
"public_key":[
    "ECDSA_224",
    "ECDSA_192",
    "RSA_2048"]
\end{verbatim}
\vspace{-15pt}
\end{small}
\end{framed}
\vspace{-10pt}
\caption{Popis algoritama poslužitelja}
\end{subfigure}
\begin{subfigure}{0.33\linewidth}
\begin{framed}
\begin{small}
\begin{verbatim}
"hash":
    "SHA-256",


"secret_key":
    "AES-CTR_256",


"public_key":
    "RSA_2048"


\end{verbatim}
\vspace{-15pt}
\end{small}
\end{framed}
\vspace{-10pt}
\caption{Dogovoreni popis algoritama}
\end{subfigure}
\caption{Primjer ponuđenih i dogovorenih algoritama}
\label{fig:neg_lists}
\end{figure}

Popisi preporuka za korištenje određenih algoritama redovito se izdaju od strane
savjetodavnih agencija za pojedine države i zajednice. Te se preporuke mogu
pronaći na web sjedištu tvrtke BlueKrypt~\footnote{BlueKrypt -- Keylength --
Cryptographic Key Length Recommendation -- \url{http://www.keylength.com/en/}}.

Protokol ACAP ne podržava mogućnost ponovne razmjene popisa
podržanih algoritama, već je potrebno ponoviti cijeli postupak dogovora kako bi
se algoritmi i ključevi osvježili i prilagodili trenutnom stanju komunicirajućih
strana. Nakon uspješnog dogovora ključeva i algoritama, aplikacija
koja se koristi protokolom definira trajanje dogovorenih parametara.
Preporučeni interval dogovora parametara je jedan sat, što je uobičajena
vrijednost trajanja IPsec \emph{Security Association} dogovora
\footnote{Trenutno strongSwan IPsec rješenje koristi preporučeni interval
od jednog sata: \\
\url{https://wiki.strongswan.org/projects/strongswan/wiki/ExpiryRekey}, dok je to kod
Cisco opreme interval od 24 sata:
\url{http://www.cisco.com/c/en/us/td/docs/security/asa/asa95/configuration/vpn/asa-95-vpn-config.pdf}}.
Taj interval može se prilagoditi ovisno o okruženju u kojem se koristi
protokol ACAP.

\section{Analiza sigurnosti i mehanizama protokola ACAP}
\label{sec:sigurnost}

Cilj protokola ACAP je omogućavanje jednostavne integracije i formalne
verifikacije kako bi se mogla jamčiti sigurna komunikacija. Zbog toga je
donekle ograničena količina funkcionalnosti kako bi se smanjila vjerojatnost
pojavljivanja napada na protokol i njegovo programsko ostvarenje.
Jedna od namjerno izostavljenih
funkcionalnosti je podrška za ponovno dogovaranje koje se nastavlja na
prethodni dogovor.  Jedino što se pamti od početnog dogovora su javni ključevi
obje komunicirajuće strane.

Poruke u protokolu sadrže samo podatke koji su nužni da bi se zaštitio dogovor
algoritama i ključeva. Kako bi se spriječili napadi ponavljanjem poruka, u svaku
poruku su uključene \emph{nonce} vrijednosti i javni dijelovi DH
razmjene. U porukama gdje nisu eksplicitno uključene one se koriste za izračun
digitalnih potpisa i HMAC sažetaka.

Princip SIGMA \cite{Krawczyk2003sigma} koristi se za sprječavanje napada
čovjek u sredini jer implicitno povezuje ključ sjednice $K$ (koji je dobiven
iz zajedničke tajne uspostavljene kroz DH razmjenu korištenjem funkcije KDF) s
javnim ključevima komunicirajućih strana $PK_I$ i $PK_R$.

Pregled sigurnosnih mehanizama i razine na kojoj se ostvaruju prikazana je u
tablici \ref{tab:zahtjevi}. Sigurnosni mehanizmi ostvareni na razini
programskog ostvarenja dodatno su objašnjeni u poglavlju \ref{ch:impl}.

\begin{table*}[hbt]
\caption{Pregled sigurnosnih mehanizama protokola ACAP}
\renewcommand{\arraystretch}{1.3}
\label{tab:zahtjevi}
\centering
\small
\begin{tabular}{p{6.4cm} p{3.0cm} p{5.0cm} }
\hline
Sigurnosni mehanizam
    & Razina rješenja
    & Način ostvarivanja
    \\ \hline \hline
Djelomični nedostatak dogovora\newline algoritma u ACAP razmjeni
    & dizajn protokola
    & mali utjecaj i mogućnost\newline dvostrukog dogovora
    \\ \hline
Računalna neovisnost tajnih ključeva
    & ostvarenje
    & ugrađen mehanizam za neovisno izračunavanje ključeva
    \\ \hline
Smanjenje utjecaja DoS napada
    & ostvarenje
    & najzahtjevniji izračuni odvojeni u pozadinski proces
    \\ \hline
Uspostavljanje povjerenja\newline komunicirajućih strana
    & dizajn protokola
    & na razini postojećih rješenja
    \\ \hline
Sprječavanje napada čovjek u sredini
    & dizajn protokola
    & verificirano u sklopu\newline formalne verifikacije
    \\ \hline
Sprječavanje napada ponavljanjem\newline poruka
    & dizajn protokola\newline i ostvarenje
    & verificirano u sklopu\newline formalne verifikacije
    \\ \hline
\end{tabular}
\end{table*}

\subsection{Djelomični nedostatak dogovora algoritama u ACAP razmjeni}
Cjelokupna razmjena sastoji se od samo četiri poruke te se uvijek radi sa svježim
\emph{nonce} vrijednostima. U slučaju da jedan od kriptografskih algoritama
koji se koristi
za razmjenu postane ranjiv, zbog malog broja poruka ne bi bilo moguće
ozbiljno ugroziti razmjenu u sklopu protokola ACAP. Zbog toga se na početku
razmjene izmjenjuje isključivo informacija o Diffie-Hellman grupi kako bi se
osiguralo dogovaranje s trenutno najboljim DH parametrima i dobila zajednička
tajna dovoljne veličine. Trenutno programsko ostvarenje protokola ACAP koristi
najnovije sigurne kriptografske algoritme, kao što su SHA-256
sažetak za HMAC izračune i kriptografija temeljena na eliptičnim krivuljama za
algoritme asimetričnog kriptosustava (DH i digitalno potpisivanje). Važno je
napomenuti da bi uvođenje jednostavnog načina dogovora algoritama za početni
dogovor zahtijevalo još barem jednu poruku na početku komunikacije i
omogućilo više točaka napada. U tom slučaju napadač bi mogao utjecati na razinu
sigurnosti tako da se koristi najslabiji trenutno podržani kriptografski
algoritam. Najbolji način izvedbe dogovora algoritma za provedbu protokola ACAP
je dvostruko korištenje protokola: jednom za dogovor algoritama za razmjenu, a
potom dogovor s novim algoritmima za konačni rezultat.

\subsection{Računalna neovisnost tajnih ključeva}
\label{sec:keys}
Ključni dio principa SIGMA leži u
računalnoj neovisnosti zajedničkih tajni koje se koriste za razmjene poruka. To
se odnosi na ključ sjednice $K$ i tajni ključ koji je rezultat cjelokupne
razmjene. Oba ključa izračunavaju se iz iste zajedničke tajne, uz uvjet da jedan
ključ ne otkriva nikakve podatke o drugom ključu, kako je opisano u
dodatku članka \cite{Krawczyk2003sigma}. Nakon što se završi razmjena protokola
ACAP, komunicirajuće strane dobivaju tajni ključ odgovarajuće duljine
za odabrani algoritam šifriranja  tijekom dogovora koji je računalno neovisan od
tajne koja je dogovorena putem protokola Diffie-Hellman. Kod programskog
ostvarenja protokola važno je da je zajednička tajna
dostupna samo u radnoj memoriji za vrijeme izvođenja dogovora i briše se odmah
nakon
razmjene kako bi se spriječilo otkrivanje podataka o tajni. U protokolu ACAP
računalna neovisnost izvedena je pravilnim korištenjem funkcije KDF za
dobivanjem računalno neovisnih ključeva.

\subsection{Smanjenje utjecaja DoS napada}
\label{sec:dos}
Napadi uskraćivanja usluge (engl.
\emph{Denial of Service, DoS attacks}) imaju za cilj iscrpiti resurse koji su na
raspolaganju određenoj usluzi kako bi se onemogućio pristup toj usluzi, što se
najčešće postiže slanjem velikog broja poruka na određenu uslugu.

Da bi 
poslužitelj stvorio poruku \initr{}, kao odgovor na poruku \initi{}, mora
odraditi
vremenski zahtjevnu operaciju generiranja nasumičnog broja pogodnog za
Diffie-Hellman razmjenu zajedno s izračunom ključa sjednice $K$ te izračunom
digitalnog potpisa i HMAC sažetka. To je računalno najzahtjevnija operacija
cjelokupne razmjene i predstavlja moguću točku napada na protokol.
Napadač bi mogao proizvesti ogromnu količinu lažnih \initi{} poruka koje bi
uzrokovale nedostatak resursa na strani poslužitelja.
Utjecaj takvog napada smanjuje se korištenjem Diffie-Hellman tajne tijekom
kratkog vremenskog perioda i prethodnim izračunom računalno zahtjevnijeg dijela
\initr{} poruke ($g^r, S_R\{g^r\}$). Generiranje zahtjevnijeg dijela poruke u
tom se slučaju obavlja kao pozadinska procedura u sklopu ACAP poslužitelja. Stoga
su jedine operacije koje poslužitelj mora
odraditi prilikom primitka poruke \initr{} izračun ključa sjednice
$K$ i izračun HMAC sažetka s tim ključem ($H_K\{PK_R, N_I, N_R\}$). Sličan
mehanizam koristi se u protokolu JFK \cite{aiello2004jfk}.

\no{
\begin{figure}[ht]
\setlength{\instdist}{4cm}
\begin{centering}
    \begin{msc}[c]{Razmjena \initi{} i \initr{} poruka}
\declinst{c}{}{Klijent (I)}
\declinst{s}{}{Poslužitelj (R)}

\mess{gen $g^r, N_R, S_R(g^{r})$}{s}{s}

\action*{generiranje $g^i,N_I$}{c}
\nextlevel[4]

\mess{$g^i, N_I$}{c}{s}
\nextlevel[1]

\action*{$K=KDF(g^{ir},N_I,N_R)$}{s}
\nextlevel[4]

\mess{$\mathbf{g^r}, \mathbf{N_R}, PK_R, \mathbf{S_R(g^r)}, H_K\{...\}$}{s}{c}

\end{msc}
\caption{Smanjenje utjecaja DoS napada}
\label{msc_dos}
\end{centering}
\end{figure}
}

\subsection{Uspostavljanje povjerenja između komunicirajućih strana}
Uspostavljanje povjerenja (engl. \emph{trust establishment}) između dvije
komunicirajuće strane predstavlja mogućnost da se na određeni način provjeri da
svakoj od komunicirajućih strana pripada adresa s pomoću koje komunicira. To
se u sklopu suvremenih računalnih sustava ostvaruje s pomoću infrastrukture
javnog ključa (engl. \emph{Public Key Infrastructure}, PKI) ili ranijom
distribucijom identifikacijskih podataka. Na taj se način stvara početno
povjerenje (engl. \emph{root of trust}) na kojem se temelji sva buduća
komunikacija.

Protokol ACAP
ne postavlja ograničenja na razmjenu identifikacijskih podataka. Mogu se
razmjenjivati samostalni javni ključevi kao i digitalno potpisani javni ključevi
u obliku certifikata. To ne zahtijeva nikakvu promjenu u porukama \initr{} i
\listi{}. U slučaju korištenja digitalno potpisanih certifikata aplikacija koja
koristi ACAP mogla bi provjeravati valjanost certifikata kako bi se osiguralo
povjerenje između komunicirajućih strana. To omogućava jednaku razinu sigurnosti
kao u protokolima SSL/TLS i IPsec te zahtijeva istu infrastrukturu javnog
ključa koja se trenutno koristi za
osiguravanje povjerenja. Ako se ne koristi PKI, onda je potrebno
odraditi prvu razmjenu protokola u kontroliranim uvjetima kako bi se zapamtili
javni ključevi i ostvarilo početno povjerenje između komunicirajućih strana. Uz
to, javne ključeve je moguće distribuirati drugim kanalom prije početne
razmjene.

\subsection{Sprječavanje napada čovjek u sredini}
U scenariju napada čovjek u
sredini napadač se klijentu u komunikaciji predstavlja kao poslužitelj i
obratno. Pretpostavka protokola ACAP je da se dogovoreni tajni ključ i algoritmi
vežu s javnim ključevima s kojima su ti parametri dogovoreni. Cilj je
onemogućiti napadača da se predstavlja kao vlasnik klijentskog odnosno
poslužiteljskog privatnog ključa, što izravno onemogućuje napadača da utječe na
dogovorene kriptografske algoritme i zajedničku tajnu u slučaju prethodno
uspostavljenog početnog povjerenja.

Pretpostavimo da
napadač pokuša promijeniti poruke u razmjeni kako bi došao do tajnog ključa koji
će dvije strane koristiti za komuniciranje. Ako napadač promijeni poruku \initi{}
($g^i,N_I$) tako da uvede drukčiji DH nasumični broj (npr. $i'$ i
izračunati $g^{i'}$),
mogao bi također izračunati povratnu \initr{} poruku korištenjem drugog para
privatnog i javnog ključa. Uz to bi morao generirati još jedan DH nasumični broj
(npr. $r'$ i izračunati $g^{r'}$) i tada bi poruka \initr{} 
bila oblika ($g^{r'}, N_R, PK_{R'}, S_{R'}\{g^{r'}\},
H_{K'}\{PK_{R'}, N_I, N_R\}$). To bi isto tako vodilo i do promjene ključa
sjednice $K$ u ključ $K'$. Sva naknadna komunikacija nakon te dvije početne
poruke bila bi neuspješna zbog provjere HMAC sažetaka i komunicirajuće strane ne
bi mogle provesti dogovor niti povezati javne ključeve s dogovorenim
kriptografskim algoritmima i promijenjenom zajedničkom tajnom. Dodatno,
otpornost na napade čovjek u sredini dokazana je postupkom formalne
verifikacije koji je prikazan u poglavlju \ref{sec:verif}.

\subsection{Sprječavanje napada ponavljanjem poruka}
\label{sec:replay}
U slučaju sprječavanja napada ponavljanjem poruka cilj napadača
je u jednom trenutku spremiti cjelokupnu razmjenu poruka u dogovoru i kasnije
odraditi identičnu razmjenu poruka kako bi komunicirajuće strane koristile
stariji ključ i kriptografske algoritme. Ovakav napad bi omogućio da se svaki
dogovor poništi tako da se ponovi stariji spremljeni dogovor.

U trenutku kad napadač šalje prethodno zabilježenu poruku, poslužitelj ima novu
vrijednost \emph{noncea} $N_R$. To znači da bi nova vrijednost trebala biti
korištena za izračun HMAC sažetka ($H_K\{PK_I, N_R, N_I\}$) koji je sadržan u
poruci \listi{}. To nije moguće odraditi u pasivnom scenariju slanja prethodno
zabilježenih poruka. U slučaju aktivnog napada napadač bi trebao biti u
mogućnosti promijeniti zabilježene poruke, što je onemogućeno promjenom
Diffie-Hellman vrijednosti za kasnije razmjene. Shodno tome sve provjere HMAC
sažetaka bile bi negativne.

\section{Usporedba protokola ACAP s protokolom za upravljanje prijenosom
podataka u protokolu SSH}

Za dogovor algoritama i ključeva u protokolu za upravljanjem prijenosom podataka
protokola SSH potrebno je šest poruka.
Prve dvije poruke (KEXINIT\textsubscript{C} i
KEXINIT\textsubscript{S}) služe za razmjenu i dogovor algoritama koji će se
koristiti za dogovor ključeva. Dogovaraju se algoritmi za Diffie-Hellman (DH)
razmjenu, algoritmi simetričnog kriptosustava i algoritmi za stvaranje zaštitnog
sažetka HMAC. Nakon dogovora algoritma razmjenjuju se poruke za dogovor
DH zajedničke tajne (KEXDH\_INIT i KEXDH\_REPLY). Poruka KEXDH\_REPLY također
sadrži javni ključ poslužitelja i digitalni potpis razmjene s kojim se
provjerava vjerodostojnost dosadašnje razmjene. Na kraju se razmjenjuju
poruke za dogovor novih ključeva (NEWKEYS\textsubscript{C} i
NEWKEYS\textsubscript{S}) nakon kojih su uspostavljeni novi ključevi za zaštitu
komunikacije. Cjelokupna razmjena poruka prikazana je na slici \ref{msc_ssh}.
Razmjenu je moguće skratiti na pet poruka tako da se poruka
NEWKEYS\textsubscript{S} šalje zajedno s porukom KEXDH\_REPLY.


Protokol za upravljanje prijenosom podataka unutar SSH mogao bi se iskoristiti
za uslugu koju pruža protokol ACAP (dogovor kriptografskih algoritama i
ključeva), ali
pritom je važno uzeti u obzir razlike između ta dva protokola. Razlike protokola
za upravljanje prijenosom podataka u odnosu na protokol ACAP su sljedeće:
\begin{itemize}
    \item veći broj poruka u razmjeni,
    \item razmjena javnog ključa samo za poslužitelj,
    \item manji broj poruka s digitalnim potpisom,
    \item kasnije otkrivanje promjene poruka za dogovor algoritama i
    \item nemogućnost smanjivanja opterećenosti poslužitelja prilikom
	generiranja digitalnog potpisa razmjene.
\end{itemize}

\begin{figure}[ht]
\setlength{\instdist}{2.5cm}
\begin{centering}
\begin{msc}[c]{Protokol SSH za upravljanje prijenosom podataka}
\declinst{c}{}{SSH Klijent (C)}
\declinst{s}{}{SSH Poslužitelj (S)}

\mess{KEXINIT\textsubscript{C}}{c}{s}
\mscmark[br]{$L_C$}{s}
\nextlevel[2]

\mess{KEXINIT\textsubscript{S}}{s}{c}
\mscmark[bl]{$L_S$}{c}
\nextlevel[1.5]

\referencestart{p}{\\$L_{CS}$}{c}{s}
\nextlevel[2]
\referenceend{p}
\nextlevel[0.5]

\action*{generiranje $x, g^x$}{c}
\action*{generiranje $y, g^y$}{s}
\nextlevel[3]

\mess{KEXDH\_INIT}{c}{s}
\mscmark[br]{$g^x$}{s}
\nextlevel[1.5]

\action*{$K=g^{xy}$}{s}
\nextlevel[2]
\action*{$DP=S_S(C, S, KEXINIT_{(S,C)}, PK_S, K)$}{s}
\nextlevel[3.5]

\mess{KEXDH\_REPLY}{s}{c}
\mscmark[bl]{$PK_S,g^y$}{c}
\nextlevel[1.5]
\action*{$K=g^{xy}$}{c}
\nextlevel[2]

\mess{NEWKEYS\textsubscript{C}}{c}{s}
\nextlevel[2]

\mess{NEWKEYS\textsubscript{S}}{s}{c}
\nextlevel[1]

\referencestart{p}{\\$K_{CS}$}{c}{s}
\nextlevel[2]
\referenceend{p}

\end{msc}
\caption{Grafički prikaz razmjene poruka protokola za upravljanje prijenosom
    podataka protokola SSH}
\label{msc_ssh}
\end{centering}
\end{figure}


Kada bi se principi iz protokola za upravljanje prijenosom podataka primijenili
za uslugu koju pruža protokol ACAP tada bi za potpunu razmjenu trebala barem
jedna poruka više. Dodatni problem je što komunicirajuće strane nisu ravnopravne
već se razmjenjuje samo javni ključ poslužitelja i ne postoji autentifikacija
klijenta u sklopu protokola za upravljanje prijenosom podataka. Autentifikacija
klijenta se odrađuje u sklopu protokola za autentifikaciju korisnika protokola
SSH. Kako se digitalni potpis koristi samo kod slanja digitalnog potpisa
razmjene složenost kriptografskih operacija tijekom razmjene je niža nego kod
protokola ACAP, ali isto tako nije moguće autentificirati klijenta.

U slučaju napada čovjek u sredini promjenu početnih poruka KEXINIT bit će moguće
otkriti tek nakon primitka KEXDH\_REPLY poruke koja sadrži digitalni potpis
razmjene. U sklopu protokola ACAP promjenu poruke u razmjeni moguće je
prepoznati prije jer su sve poruke osim prve zaštićene zaštitnom sumom HMAC ili
digitalnim potpisom.  Konačno digitalni potpis razmjene sadrži podatke koje je
poslao klijent, te ga je potrebno generirati u trenutku kad poslužitelj primi i
poruku KEXINIT\textsubscript{C} i KEXDH\_INIT. 
Zbog toga je protokol SSH osjetljiviji na napad uskraćivanja usluge u odnosu na
ACAP.

% vim: spell spelllang=hr
